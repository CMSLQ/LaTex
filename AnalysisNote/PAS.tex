\documentclass{cmspaper}
\usepackage{lineno}
\begin{document}
\begin{linenumbers}

\begin{titlepage}

\cmspas{EXO-08-010}
%  \internalnote{2005/000}
%  \conferencereport{2005/000}

%\begin{center}
%{\bf \huge  CMS Physics Analysis Summary }
%\end{center}

\date{\today}

\title{Search for First Generation Scalar Leptoquarks at the CMS Experiment}


  \begin{Authlist}
%   Sarah Eno, Paolo Rumerio, Francesco Santanastasio, Elizabeth Twedt
    The CMS Collaboration, CERN
    %\Instfoot{cern}{University of Maryland, College Park, MD, USA}
  \end{Authlist}


  \begin{abstract}
    The discovery potential for pair production of first generation scalar leptoquarks 
    decaying to an electron and quark is investigated 
    at the CMS experiment using Monte Carlo samples produced with a full simulation of the detector response.  
    The analysis strategy is defined assuming an available integrated luminosity of 100 pb$^{-1}$.
    Event selection, and reconstruction and identification of high energy electrons and jets are discussed.
    The electron-jet invariant mass distribution is reconstructed
    and used to identify the possible presence of a leptoquark signal.
    Data-driven techiques are discussed to determine the main Standard Model backgrounds using 
    control samples. The signal yield is extracted using a maximum likelihood fit to the 
    electron-jet invariant mass distribution, and the significance of the measurement is estimated.
    Finally the CMS discovery potential is quantified for different leptoquark mass hyphotheses.

%The leptoquarks are assumed to decay exclusively to electrons and quarks.  Detailed explanations of the discovery techniques and data driven methods to extract the signal from background events are given.  
%Both exclusion limits and discovery potential are discussed for masses of the leptoquark ranging from around the current limit of 251 GeV to 1 TeV.
  \end{abstract} 

\end{titlepage}

%=================================================================
\setcounter{page}{2}%JPP

\section{Introduction}
 
This note describes the analysis techniques that will be used to search for evidence of first generation scalar leptoquarks 
at the CMS experiment~\cite{CMSTDR1},\cite{CMSTDR2}. The methods described are evaluated using Monte Carlo (MC) simulations assuming an availability of 
100 pb$^{-1}$ of integrated luminosity and 7 TeV proton beams.

Leptoquarks (LQ) are conjectured exotic particles with both baryon and lepton number, allowing their decay into a lepton and quark.  
Several models of new physics beyond the Standard Model (SM) predict the existence of leptoquarks, including General Unified Theories, 
technicolor and composite models \cite{theories}.  

The model parameters are 1) $M_{LQ}$, the LQ mass, 2) $\beta$, the branching ratio of the decay 
$\mbox{LQ} \rightarrow l + q$, where $l$ is a charged lepton and $q$ is a quark, 
3) $\lambda_{\mbox{LQ}lq}$, the coupling between LQ, lepton and quark ($\mbox{LQ}-l-q$ vertex). 
The results of the H1 experiment at HERA show that the data are in agreement with a scenario of weak 
$\lambda_{\mbox{LQ}lq}$ coupling \cite{hera}. The choice here adopted
is to fix the paramameter $\lambda_{\mbox{LQ}lq} = \lambda_{EM} \approx 0.3$, which gives 
a theoretical LQ relative width ($\Gamma_{LQ}/M_{LQ}$) around 0.2\%. 
In this scenario the measurement of $\Gamma_{LQ}/M_{LQ}$ cannot be performed at LHC, since it is completely dominated by the detector resolution.  

Experiments at the Tevatron have placed the current lower limit on the mass of first generation scalar 
leptoquarks around 290 GeV (assuming $\beta=1$ in the electron channel) \cite{d02008}.
At the LHC, the dominant production modes of leptoquarks will be through gluon-gluon fusion and 
quark-antiquark annihilation producing a leptoquark-antileptoquark pair~\cite{Kramer}. 
The cross section of these two processes is almost independent from the value of 
$\lambda_{\mbox{LQ}lq}$, since there is no $\mbox{LQ}-l-q$ vertex in the Feynman diagram 
at leading order. This note studies the signal of first generation scalar leptoquarks decaying into electron and 
(light) quark with a branching ratio $\beta=1$. 
For this scenario, the experimental signature of a LQ event is striking: two 
high transverse momentum electrons and two high transverse momentum jets 
whose combination gives a peak in the electron-jet invariant mass 
spectrum corresponding to the LQ mass. No true missing transverse energy is expected in LQ events.

\section{Monte Carlo Samples} \label{sec:MCSamples}
MC signal samples are generated with leptoquark masses ranging from 250 GeV to 1 TeV and $\beta=1$. 
In the mass range investigated the dominant process is gluon-gluon fusion. 
The Full Simulation (FullSim) signal samples are produced using 
the official CMS software for generation (using PYTHIA~\cite{PYTHIA}), 
simulation (based on GEANT4~\cite{GEANT4}), digitization and reconstruction. 
These signal samples have been produced with a 100 $\mbox{pb}^{-1}$ mis-calibration and mis-alignment 
scenario of the detector performances. Table~1
%\ref{tab:NumEvents} 
shows the cross sections at leading order (LO) of signal samples generated at different LQ mass.

\begin{table}[htb]
  \label{tab:NumEvents}
  \begin{center}
    \begin{tabular}{|l|c|} \hline
%      & \multicolumn{4}{c|}{Leptoquark Mass (GeV)} \\ 
      LQ mass (GeV) & $\sigma_{LO}$ (pb) \\ \hline
      250  & 14 \\
      400  & 1.2 \\
      650  & 0.076 \\
      1000 & 0.0047 \\
      \hline
    \end{tabular}
    \caption{\small \sl Cross sections at leading order (LO) of signal samples generated at different LQ mass.}
  \end{center}
\end{table}

The SM background samples are taken from the CMS official production of year 2007, 
which consists of more than 150M events of various SM processes roughly representing the first 100 pb$^{-1}$ of LHC data. 
The background samples used for this analysis are listed below:
\begin{itemize}
%
\item $t\bar{t}$+Njet events (where N=0,1,..,4), generated with ALPGEN \cite{Mangano:2002ea} inclusive production (all decays), LO-NLO K factor applied 
in the cross section; 
%
\item $Z/\gamma$+Njet events (where N=0,1,..,5), generated with ALPGEN, $ 0 < P_{T}^{Z/\gamma} < 300 $ GeV and $40<M_{Z/\gamma}<200$ GeV, 
$Z/\gamma*$ decaying into charged leptons;  
%
\item $W$+Njet events (where N=0,1,..,5), generated with ALPGEN, $ 0 < P_{T}^{W} < 300 $ GeV, $W$ decaying into leptons;  
%
\item QCD multi-jet events, generated with PYTHIA, inclusive production in $P_{T}^{jet}$ bins from 0 to $\infty$;  
%
\item $\gamma$+jet events, generated with PYTHIA, $ 0 < P_{T}^{jet} < 7000 $ GeV;  
%
\item $WW$, $WZ$, $ZZ$ events, generated with PYTHIA, inclusive production (all decays).
\end{itemize} 

\section{Trigger Studies} \label{sec:trig}

The presence in the final state of two high energy electrons is exploited for the online selection 
of candidate leptoquark-antileptoquark events. 
Three electromagnetic High Level Trigger (HLT) paths are investigated
% \footnote{In the most recent HLT trigger table these names have been changed to $HLT\_EM80$, $HLT\_EM200$, 
%and $HLT\_IsoEle18\_L1R$.}
%In the most recent HLT trigger table these names have been chaged to HLT_EM80, HLT_EM200, and HLT_IsoEle18_L1R}
:
\begin{itemize}
\item the EMHighEt trigger (e.m. energy deposit with $E_{T}>80$ GeV + very loose isolation criteria);
\item the EMVeryHighEt  trigger (e.m. energy deposit with $E_{T}>200$ GeV + no isolation criteria);
\item the SingleElectronRelaxed trigger (e.m. energy deposit with $E_{T}>18$ GeV + track matching + relaxed isolation criteria).
\end{itemize}

The EMHighEt and EMVeryHighEt triggers have been specifically designed in order to have a high 
efficiency for electrons and photons with high transverse energy\cite{HLT_HE_VHE}. 

The efficiencies for leptoquark events of the considered triggers have been studied with the FullSim 
samples of Section~\ref{sec:MCSamples} at four leptoquark mass points and are reported in 
Fig.~\ref{fig:HLTeffic}.
In order to guarantee a high efficiency and at the same time contain the data sample size, the current decision is to perform the analysis on the OR of the EMHighEt and EMVeryHighEt triggers.

\begin{figure}[htbp]
  \begin{center}
  \begin{tabular}{cc}
  \resizebox{12cm}{!}{\includegraphics{plots/triggerEffic.eps}} 
  \end{tabular}
  \caption{\small \sl High Level Trigger efficiencies for LQ events at different mass. 
    Single triggers and their $OR$ are shown.}
    \label{fig:HLTeffic}
  \end{center}
\end{figure}

\section{Reconstructed Objects} 

\subsection{Electron Studies} \label{sec:electrons}
An electron reconstructed object is formed from a deposit of energy in ECAL (called ``supercluster'') with a track pointing towards it.
In this study the standard ``PixelMatchGsfElectrons'' collection is used~\cite{CMSTDR1}.
Standard corrections are applied to set correctly the electromagnetic (e.m.) energy scale.

The criteria for electron identification (ID) at the reconstruction level are very loose, resulting in several fake electrons in the collection. 
In this analysis the HEEP selection for electron ID and isolation \cite{HEEP}, which is optimized for 
electrons with an energy of hundreds of GeV, have been applied in an effort to 1) keep the efficiency high on electrons with high $P_{T}$ from LQ decays, and 
2) reduce the number of jets faking electrons in the collection.
The definitions of the variables are the following:
%
\begin{itemize}
%
\item $H/E$: the ratio of the hadronic energy of all the HCAL Rec Hits in a cone of $\Delta R=0.1$\footnote{$\Delta R = \sqrt{\Delta\phi^2 + \Delta\eta^2}$}, 
centered on the electron's position in the calorimeter, 
to the electromagnetic energy of the supercluster associated to the electron;
%
\item $\sigma_{\eta\eta}$: the shape variable, which describes the electron shower size along the $\eta$ direction, is 
defined as 
\begin{displaymath}
\sigma_{\eta\eta} = \sum_{i \in \mathrm{crystals}} ( \eta_i - \eta_s )^2 \frac{E_i}{E_{\mbox{seed cluster}}} \quad ,
\end{displaymath}
where $i$ is the index of a crystal, and $s$ is the index of the seed cluster of the electron's supercluster;
%
\item $\Delta\eta_{in}$ and $\Delta\phi_{in}$: the difference in $\eta$ and $\phi$ between the track position as 
measured in the inner layer tracker, extrapolated to the ECAL surface, and the $\eta$ and $\phi$ of the supercluster;
%
\item number of tracks: with $P_{T}>1.5$ GeV in a cone of $0.1 < \Delta\mbox{R} < 0.2 $ around the electron direction;
%
\item track isolation: the sum of the $P_{T}$ of the tracks (as defined above) divided by the $E_{T}$ of the supercluster;
%
\item e.m. isolation: the transverse electromagnetic energy of all the clusters in a cone $\Delta\mbox{R} < 0.3$ centered on the 
electron's position in the calorimeter, excluding the clusters that make up the supercluster;
%
\item hadronic isolation: the transverse hadronic energy of all the HCAL Rec Hits with an $E_{T}>0$ in a cone of
$0.15 < \Delta\mbox{R} < 0.3$ centered on the electron's position in the calorimeter. 
%
\end{itemize}
Table~2
%\ref{tab:HEEPselection} 
summarizes the cuts from HEEP selections which are 
used in this analysis. In addition only reconstructed electrons with $P_{T}>30$ GeV and within the ECAL fiducial region 
$|\eta|<1.4$ + $1.6<|\eta|<2.0$ are included in the electron collection used in the analysis.  

%Recentely the HEEP selection criteria have been sligthly improved on e.m. and hadronic isolation, 
%leading to a better rejection of jets faking electrons, and keeping the same efficiency on real electrons.
%The impact of this improved selection criteria on the results here presented will be investigated in future upgrades of 
%this analysis. 

\begin{table}[htbp]
  \label{tab:HEEPselection}
  \begin{center}
    \begin{tabular}{|lcc|lcc|} \hline
      \multicolumn{3}{|c|}{ID Variables} & \multicolumn{3}{|c|}{Isolation Variables} \\ 
      Variable & Barrel & Endcap & Variable & Barrel & Endcap  \\ \hline
      $H/E$  & $<0.05$ & $<0.08$ & number of tracks  & $<4$ & $<4$ \\ \hline
      $\sigma_{\eta\eta}$  & $<0.011$ & $<0.0275$ & track isolation (GeV) & $<7.5$ & $<15$ \\ \hline
      $|\Delta\eta_{in}|$  & $<0.005$ & $<0.007$ & e.m. isolation (GeV) & $<6+0.01*E_{t}$ & $<6+0.01*E_{t}$ \\ \hline
      $|\Delta\phi_{in}|$  & $<0.09$ & $<0.09$ & hadronic isolation (GeV) & $<4+0.005*E_{t}$ & $<4+0.005*E_{t}$ \\ \hline
    \end{tabular}
  \caption{\small \sl HEEP electron identification and isolation selection criteria}
  \end{center}
\end{table}

Electrons coming from leptoquark decays, 
being typically isolated from jets and having a $P_{T}$ of hundreds of GeV can be easily reconstructed with high efficiency. 
The electron acceptance in the fiducial region ($A_{FR}$), the 
reconstruction efficiency within the fiducial region ($\varepsilon_{FR}^{reco}$), and the overall reconstruction efficiency 
($A_{FR} \times \varepsilon_{FR}^{reco}$) for single electrons coming from LQ decays at different LQ mass is shown in Table~3
%\ref{tab:ElecEffAcc}
.
The overall reconstruction efficiency on a single electron, including the acceptance cuts, is around $70\%$ for LQ masses between 250 GeV and 1 TeV.

\begin{table}[htb]
  \label{tab:ElecEffAcc}
  \begin{center}
    \begin{tabular}{|l|c|c|c|} \hline
      LQ mass (GeV) & $A_{FR}$ & $\varepsilon_{FR}^{reco}$ & $A_{FR} \times \varepsilon_{FR}^{reco}$\\ \hline
      250 & 80.8\% & 88.1\% & 71.2\% \\ \hline
      400 & 84.4\% & 87.1\% & 73.5\% \\ \hline
      600 & 87.3\% & 86.6\% & 75.6\% \\ \hline
      1000 & 89.2\% & 83.1\% & 74.1\% \\ \hline
    \end{tabular}
    \caption{\small \sl Electron acceptance in the fiducial region ($A_{FR}$), the 
      reconstruction efficiency within the fiducial region ($\varepsilon_{FR}^{reco}$), and the overall reconstruction efficiency 
      ($A_{FR} \times \varepsilon_{FR}^{reco}$) for single electrons coming from LQ decays. Efficiency are 
      calculated using FullSim samples at different LQ mass. Statistical relative errors on efficiency are less than 2\%.}
  \end{center}
\end{table}

\subsection{Jet Studies} \label{sec:jet}

The reconstruction jet algorithm used in this analysis is an iterative cone algorithm with a radius of $\Delta R=0.5$ \cite{JetAlg}.  
In order to remove the electron contamination from the jet collection (disambiguation) a simple procedure is applied.  The collection 
of electrons passing the ID selection criteria is created, 
as described in Section \ref{sec:electrons}, and a jet is dismissed if it is within 
a $\Delta R$ of 0.5 from one of the two leading electrons. 

The corrections for the Jet Energy Scale (JES) are divided into various levels \cite{JES}.  
In this analysis the jet energy is corrected with the level 2, 3, and 5 corrections.  
The level 2 jet energy corrections are the 
``relative jet corrections'', and correct the energy in order to have a uniform response in psudorapidity.  
The level 3 corrections are the ``absolute jet corrections''.  These aim at correcting the jet energy in the 
region of $|\eta| < 1.3$ back to the particle level energy, which would correspond to the generator jet energy in the MC samples. 
The level 2 and 3 corrections are the standard corrections recommended for all analysis in CMS. 
In addition, level 5 corrections are applied to take into account the flavor of the parton from which the jet originated.
This is because the L2 and L3 corrections are for a mixture of quark and gluon jets, while LQ's 
have just quark jets. The total corrections increase the jet energy by $\approx 20\%$ on average.

%The level 2 and 3 corrections are calculated for samples enriched in gluon-jets such as QCD multi-jet events. Jets from 
%first generation LQ decays, however come only from light quarks, instead of a mixture of quarks and gluons.  Since the gluon-jets have
%a softer fragmentation function, the level 2 and 3 corrections tend to overcompensate the loss of energy when applied to jets originated 
%from quarks. 

\section{Event Selection} \label{sec:eventSelection}
The basic strategy to identify the existence of a particle that decays to a jet and an electron 
is to study the invariant mass of the electron-jet pairs in the events. 
The signal of a leptoquark would appear as a bump in the distribution of this invariant mass.
The following selections have been made in order to minimize the number of background events 
selected while retaining a high signal efficiency.

The online selection of candidate events is made by the HLT triggers EMHighEt and EMVeryHighEt 
described in Section~\ref{sec:trig} while the offline identification of electrons and jets 
proceeds as described in Sections~\ref{sec:electrons} and \ref{sec:jet}, respectively.
The offline selection of the eejj sample continues with the following cuts:

\begin{enumerate}
\item at least 2 isolated electrons, one required to have $P_T>85$~GeV, 
and the other $P_T>30$~GeV 
\item at least 2 jets, both required to have $P_T>50$~GeV
\item $M_{ee}>100$~GeV
\item $S_T\equiv P_T(e_1)+P_T(e_2)+P_T(j_1)+P_T(j_2)>400$~GeV
\end{enumerate}

The $P_T$ threshold for one electron in cut 1 is set to 5~GeV above the threshold of the EMHighEt trigger 
in order to increase the confidence that the trigger was actually fired by a real electron in all events.
The other electron will have a $P_T>30$~GeV as it is implicit in the definition of electron. 
Cut 3, where $M_{ee}$ is the invariant mass of the electron pair, removes background events from 
$Z/\gamma$+jets events as shown in Figure~\ref{fig:Mee_St_distributions}-left.
Cut 4, where the variable $S_T$ is defined as the scalar sum of the transverse momenta of the 
2 electrons and 2 jets with highest $P_{T}$ of the event, 
is applied following the approach of the experiment $D0$ in 
\cite{Abazov:2001mx}. In that paper, an event selection optimization as a function of
the combinations of several kinematic variables has shown that $S_T$ is the most powerful one 
and little is gained by adding cuts on other variables. The distribution of $S_T$ for the present
analysis is shown in Figure~\ref{fig:Mee_St_distributions}-right.


Once the eejj sample is built, there are two ways to combine two electrons and two jets to make two electron-jet pairs. 
For each event, the combination with the minimum difference, $\Delta M_{ej}$, between the invariant masses, $M_{ej}$, 
of the two electron-jet pairs is chosen. 
The resulting $M_{ej}$ distribution is shown in Figure~\ref{fig:Mej_allComb} for the signal and remaining backgrounds. 
A detailed study of the optimization of the selection criteria, and of alternative algorithms to identify the best candidate 
electron-jet combinations will be performed in future upgrades of this analysis.

\begin{figure}[htbp]
  \begin{center}
    \begin{tabular}{cc}
      \resizebox{7.5cm}{!}{\includegraphics{plots/Mee_distribution.eps}} &
      \resizebox{7.5cm}{!}{\includegraphics{plots/St_distribution.eps}} \\
    \end{tabular}
    \caption{\small \sl Left: invariant mass of the electron pair, $M_{ee}$. 
             Right: scalar sum of the $P_T$ of the 2 leading electrons and 2 leading jets. 
	     In each histogram, the distributions for the signal (for leptoquark mass 250 and 650~GeV) and the 
	     relevant backgrounds are shown after applying all cuts except the one involving the 
	     plotted variable. 
	     The background histograms are summed on top of each other.
	     The histogram labeled with ``Others'' contains: 1) for $M_{ee}$, $W$+jets (33\%), 
	     $WW$ (11\%), $ZZ$ (20\%), and $WZ$ (36\%) events; 2) for $S_{T}$, $W$+jets (57\%), 
	     $WW$ (19\%), $ZZ$ (8\%), $WZ$ (14\%), and $\gamma$+jet (2\%) events.
	     With the available MC statistics, no QCD multi-jet events pass 
	     the selection.}
	     %Arrows indicate the cut subsequently applied.
    \label{fig:Mee_St_distributions}
  \end{center}
\end{figure}


\begin{figure}[htbp]
  \begin{center}
    \resizebox{10cm}{!}{\includegraphics{plots/Mej_best_Meecut_Stcut.eps}}
    \caption{\small \sl Distribution of the invariant mass, $M_{ej}$, of the electron-jet pairs 
      with smaller $\Delta M_{ej}$
      for signal (at 250~GeV and 650~GeV LQ mass) and backgrounds. 
      The complete event selection has been applied.
      The background histograms are summed on top of each other.
      The histogram labeled with ``Others'' contains $W$+jets (65\%), 
      $WW$ (20\%), $ZZ$ (8\%), and $WZ$ (7\%) events. 
      With the available MC statistics, no QCD multi-jet and $\gamma$+jet events pass 
      the selection.}
    \label{fig:Mej_allComb}
  \end{center}
\end{figure}

The efficiency of each selection cut is shown in Table \ref{tab:selection_effic_250} 
for a LQ sample with a mass of 250 GeV.  
Table~\ref{tab:selection_effic_ttbar} shows the number of events selected by each cut 
for $t\bar{t}$ and $Z/\gamma$+jet events, which are the dominant backgrounds in the eejj sample. 
A summary of the number of selected signal and background events expected in 100 pb$^{-1}$ of data 
is reported in Table \ref{tab:EventSelSummary}. 
The overall signal selection efficiencies 
are around 30-50\% for the LQ masses investigated. 
After the described event selection, the dominant background 
contributions are $t\bar{t}$ and $Z/\gamma$+jets. 

\begin{table}[htbp]
\begin{center}
\begin{tabular}{|c|c|c|}
\hline
\hline
 & $N_{ev}$ $100pb^{-1}$ & $\varepsilon$ \\
\hline
\hline

no cut &1400 $\pm$ 44& - \\
+ hlt (HE or VHE) &1121 $\pm$ 40 & 8.01e-01 $\pm$ 1.26e-02\\
+ 2 ele (ID) $P_{T} >30$ GeV &644 $\pm$ 30 & 5.74e-01 $\pm$ 3.36e-02\\
+ 2 ele (ID+Iso) $P_{T} >30$ GeV &592 $\pm$ 29 & 9.20e-01 $\pm$ 6.19e-02\\
+ 2e, 2j $P_{T}^{ele}$ ($P_{T}^{jet}$) $>$30(50) GeV &508 $\pm$ 27& 8.58e-01 $\pm$ 6.14e-02\\
+ $P_{T}$ $1^{st}$ ele $>$ 85 GeV &506 $\pm$ 27& 9.97e-01 $\pm$ 7.41e-02\\
+ $M_{ee} >100$ GeV& 455 $\pm$ 25& 8.98e-01 $\pm$ 6.86e-02\\
+ $S_{T} >400$ GeV &425 $\pm$ 24& 9.35e-01 $\pm$ 7.46e-02\\
\hline

full selection efficiency& & 3.04e-01 $\pm$ 1.45e-02\\
\hline
\end{tabular}
\end{center}
\caption{\small \sl LQ sample with $M_{LQ}=250$ GeV: the first column lists the selection sequence, $N_{ev}$ $100pb^{-1}$ is the number of selected events in $100pb^{-1}$, $\varepsilon$ is the relative selection efficiency with respect to the previous cut. The full selection efficiency is shown in the last line.}
\label{tab:selection_effic_250}
\end{table}



\begin{table}[htbp]
\begin{center}
\begin{tabular}{|c| |c|c|}
\hline
\hline
 & $t\bar{t}$ sample  & $Z/\gamma$ sample\\
 & $N_{ev}$ $100pb^{-1}$ & $N_{ev}$ $100pb^{-1}$ \\
  
\hline
\hline

hlt (HE or VHE) &2115.9 $\pm$ 9.6 & 2466 $\pm$ 14 \\
+ 2 ele (no Iso) pT $>30.0$ GeV &137.4 $\pm$ 2.5& 1356 $\pm$ 12 \\
+ 2 ele (Iso) pT $>30.0$ GeV &105.4 $\pm$ 2.1& 1329 $\pm$ 12  \\
+ 2e, 2j pT$_{ele}$(pT$_{jet}$) $>$30.0(50.0) GeV &49.6 $\pm$ 1.5& 124.7 $\pm$ 3.0\\
+ $P_{T}$ $1^{st}$ ele $>$ 85 GeV &47.2 $\pm$ 1.4& 115.7 $\pm$ 2.9\\
+ $Mee >100.0$ GeV&40.2 $\pm$ 1.3& 10.82 $\pm$ 0.89 \\
+ $St >400.0$ GeV &21.26 $\pm$ 0.98 & 5.10 $\pm$ 0.60 \\
\hline
\end{tabular}
\end{center}
\caption{\small \sl $t\bar{t}$ and $Z/\gamma$ samples: the first column lists the selection sequence, $N_{ev}$ $100pb^{-1}$ is the number of selected events in $100pb^{-1}$.}
\label{tab:selection_effic_ttbar}
\end{table}



%\begin{table}[htbp]
%\begin{center}
%\begin{tabular}{|c| |cc| |cc|}
%\hline
%\hline
% & $t\bar{t}$ sample &              & $Z/\gamma$ sample &\\
% & $N_{ev}$ $100pb^{-1}$ & $N_{ev}$ &  $N_{ev}$ $100pb^{-1}$ & $N_{ev}$ \\
%  
%\hline
%\hline
%
%hlt (HE or VHE) &2115.9 $\pm$ 9.6& 48359.0 & 2466 $\pm$ 14 & 30413 \\
%+ 2 ele (no Iso) pT $>30.0$ GeV &137.4 $\pm$ 2.5& 3139.0 & 1356 $\pm$ 12 & 13136 \\
%+ 2 ele (Iso) pT $>30.0$ GeV &105.4 $\pm$ 2.1& 2408.0 & 1329 $\pm$ 12 & 12830 \\
%+ 2e, 2j pT$_{ele}$(pT$_{jet}$) $>$30.0(50.0) GeV &49.6 $\pm$ 1.5& 1117.0 & 124.7 $\pm$ 3.0& 1807 \\
%+ $P_{T}$ $1^{st}$ ele $>$ 85 GeV &47.2 $\pm$ 1.4& 1063.0 & 115.7 $\pm$ 2.9& 1694 \\
%+ $Mee >100.0$ GeV&40.2 $\pm$ 1.3& 1013.0 & 10.82 $\pm$ 0.89 & 157 \\
%+ $St >400.0$ GeV &21.26 $\pm$ 0.98 & 476 & 5.10 $\pm$ 0.60 & 76 \\
%\hline
%\end{tabular}
%\end{center}
%\caption{\small \sl $t\bar{t}$ and $Z/\gamma$ samples: the first column lists the selection sequence, $N_{ev}$ $100pb^{-1}$ is the number of selected events in $100pb^{-1}$, $N_{ev}$ is the absolute number 
%of selected events (unweighted for the cross section).}
%\label{tab:selection_effic_ttbar}
%\end{table}
%

\begin{table}[htbp]
\begin{center}
\begin{tabular}{|cccc||ccc|}
\hline
                 &   Signal&         &          & & Background & \\
$M_{LQ}$=250 GeV & 400 GeV & 650 GeV & 1000 GeV & $t\bar{t}$  & $Z/\gamma$+jet & Others \\
\hline
425 $\pm$ 24 & 52.4 $\pm$ 2.5 & 3.4 $\pm$ 0.1 & 0.21 $\pm$ 0.01 & 21.3 $\pm$ 1.0 & 5.1 $\pm$ 0.6 & 1.1 $\pm$ 0.3 \\
\hline
\end{tabular}
\end{center}
\caption{\small \sl Number of selected events in the eejj sample expected in 100 pb$^{-1}$ of data for signal events at different LQ masses 
and for background events. 
The LQ cross section rapidly falls at high LQ mass, thus 
producing a relative decrease in the number of selected events. 
In the background part, ``Others'' includes $W$+jet, and 
di-boson ($WW$,$WZ$,$ZZ$) events. 
With the available MC statistics, no QCD multi-jet and $\gamma$+jet events pass the final selection.}
\label{tab:EventSelSummary}
\end{table}


\section{Data-driven techniques for background estimate} \label{sec:bkgStudy}
This section describes data-driven techniques used to estimate the shape of the $M_{ej}$ distribution for $t\bar{t}$ and $Z/\gamma$+jet backgrounds using 
control samples. These shapes are used in a fit to the data to extract the number of signal events, as described 
in Section~\ref{sec:signalExtraction}. The methods for background estimation described here will not rely on the Monte Carlo 
once real data is available. They are
therefore particularly suited for first data taking when the confidence that the MC describes well the data is expected to be limited.
\subsection{$t\bar{t}$ background shape}
A good control sample can be obtained by using the same selection criteria applied for the eejj sample, but 
requiring at least one electron and one muon (e$\mu$jj sample) instead of 2 electrons in the final state, in addition to the two jets. 
For $t\bar{t}$ events, the $M_{lj}$ distributions of the e$\mu$jj and the eejj samples
are expected to be very similar in shape since the kinematics of the process does not depend 
on the nature of the lepton. 
 Figure~\ref{fig:ttbar} shows a good agreement between 
the shape of the two distributions with the current MC statistics available. 
With 100 pb$^{-1}$ of data the e$\mu$jj sample is expected to have about 30 events ($\approx 60$ entries in the 
$M_{lj}$ distribution). 
The e$\mu$jj sample is enriched in $t\bar{t}$ events with a small contamination of about 
2\% of $WW$ and $Z/\gamma$+jet events. 

%For a sample of $t\bar{t}$ events at generator level, the number of e$\mu$jj events is expected to be exactly two times the number of 
%eejj events, considering all the possible combinations of the $W$ decays. The trigger filter, the offline selection, 
%and the different reconstruction efficiency, acceptance and $P_{T}$ resolution between electron and muon
%put a bias in the relative amount of eejj and e$\mu$jj events.
%In general it is possible to correct for these effects in order to estimate the number of eejj in the signal sample directly from 
%the size of the e$\mu$jj control sample. In this way the control sample can be used to estimate both the normalization and the shape
%of the $t\bar{t}$ background. In this analysis only the shape of the e$\mu$jj sample is used in the signal extraction, but
%the possibility to get also the normalization from the control sample will be investigated in future upgrades of this analysis.

\begin{figure}[htb]
  \begin{center}
  \begin{tabular}{cc}
  \resizebox{10cm}{!}{\includegraphics{plots/Mlj_eejj_emujj_ttbarControlSample.eps}} \\ 
  \end{tabular}
  \caption{Distributions of the lepton-jet invariant mass for the eejj and the e$\mu$ samples, for $t\bar{t}$ events}
  \label{fig:ttbar}
  \end{center}
\end{figure}

\subsection{$Z/\gamma$+jet background shape} 

The same concept of control sample can be applied to estimate the $M_{ej}$ shape of the $Z/\gamma$+jet background.
In this case the control sample can be obtained by using the same selection criteria applied for the eejj sample except the $M_{ee}$ cut, which is modified to select events with a real $Z$ boson reconstructed ($ 80\mbox{ GeV} < M_{ee} < 100\mbox{ GeV}$). 
This control sample is an almost pure sample of  
$Z/\gamma$+jet events (about 4\% contamination dominated by $t\bar{t}$ and $WZ$ events) since the cross section of the process is resonant at the Z mass, and is 
independent from the signal sample by construction. In addition the control sample has about 10 times the number of $Z/\gamma$+jet events in 
the signal sample (about 50 events expected in the control sample for 100 pb$-1$ of data), which 
is important to reduce the statistical errors in the determination of the background shape.

Figure~\ref{fig:zjet} shows the good agreement between the shape of the $M_{ej}$ distribution for the two samples 
with the current MC statistics available. The agreement observed can be interpreted as the consequence of the 
weak correlation between the two reconstructed quantities $M_{ej}$ and $M_{ee}$ for selected events. Indeed, it is observed that the selected electrons acquire a high $P_{T}$ due to
the substantial boost, rather than a high mass, of the $Z/\gamma$ bosons.  

\begin{figure}[htb]
  \begin{center}
  \begin{tabular}{cc}
  \resizebox{10cm}{!}{\includegraphics{plots/Mej_eejjIN_eejjOUT_zjetControlSample.eps}} \\ 
  \end{tabular}
  \caption{Distributions of the electron-jet invariant mass for the signal eejj sample 
    (cut 1,2,3,4)
    and the control sample (cut 1,2,4 + $80\mbox{ GeV} < M_{ee} < 100\mbox{ GeV}$), for $Z/\gamma$+jet events.}
  \label{fig:zjet}
  \end{center}
\end{figure}

\section{Signal Extraction} \label{sec:signalExtraction}

This section describes the procedure and verifies the ability to extract the 
number of signal events from a distribution of the invariant mass, $M_{ej}$, of the 
electron-jet pairs with the characteristics expected from the studies 
of the previous sections.

MC experiments are generated to produce $M_{ej}$ distributions that are the sum
of a leptoquark signal and the only relevant backgrounds: $t\bar{t}$ and $Z/\gamma$+jets.
The average leptoquark signal distribution, $h_s$, is taken from the FullSim MC 
sample at the leptoquark mass under consideration after applying the selection
described in Section~\ref{sec:eventSelection}.
The average distributions, $h_{t\bar{t}}$ and $h_{Z/\gamma\mathrm{+jets}}$ 
of the $t\bar{t}$ and $Z/\gamma$+jets backgrounds are determined from the control samples
and procedures described in Section~\ref{sec:bkgStudy}.

A log-likelihood fit is performed on the histogram of each generated MC experiment
using a function
\begin{displaymath}
  h_{s+b} = N_s \cdot \tilde{h}_s + N_b \cdot \tilde{h}_b
\end{displaymath}
where $N_s$ and $N_b$ are the free parameters of the fit for the number of signal and background events,
\begin{displaymath}
  \tilde{h}_b = \frac{R}{1+R} \tilde{h}_{t\bar{t}} + \frac{1}{1+R} \cdot \tilde{h}_{Z/\gamma\mathrm{+jets}}~\mathrm{,}
\end{displaymath}
and $\tilde{h}_x$ represents a general distribution $h_x$ normalized to unity. 

The ratio
\begin{displaymath}
  R \equiv N_{t\bar{t}} / N_{Z/\gamma\mathrm{+jets}}~\mathrm{,}
\end{displaymath}
between the number of $t\bar{t}$ and $Z/\gamma$+jets events is currently determined from 
MC and has a value $R_{MC}=4.2$.
Figure~\ref{fig:Mej_fit} shows a MC experiment and the result of the fit.
%A tendency of the fit to underestimate the number of signal events is observed. 
%Such an effect amounts to about 10\% with a moderate dependency on the leptoquark mass and the 
%integrated luminosity. This effect may bias the results towards an overestimation of the integrate%d luminosity needed for discovery.

The fitting procedure described in this section is used in the following section
to determine the CMS potential for discovering a leptoquark signal. 


 \begin{figure}[htb]
   \begin{center}
     \resizebox{10cm}{!}{\includegraphics{plots/Mej_fit.eps}}
     \caption{\small \sl A MC experiment (dots with error bars) generated for a leptoquark mass of 
       400~GeV (blue-filled histogram), a sum of the $t\bar{t}$ and $Z/\gamma$+jets 
       backgrounds (gray-filled histogram) with a relative ratio R=4.2 determined from 
       full simulation MC, and a number of signal events $N_s=50$ corresponding
       to an integrated luminosity of 95~$pb^{-1}$.
       The blue-line open histogram is the result of the fit and the estimate of signal 
       events is 48.5$\pm$ 3.8.}
     \label{fig:Mej_fit}
   \end{center}
 \end{figure}



\section{CMS Potential} \label{CMSpotential}

In order to estimate the integrated luminosity needed by CMS to claim 
a discovery in the electron channel of the first generation leptoquark, 
several MC experiments with signal and background events are produced
as described in Section~\ref{sec:signalExtraction}.

The significance of each experiment is extracted using the log-likelihood estimator 
\begin{displaymath}
S_L \equiv \sqrt{2\cdot ln{(L_{s+b}/L_{b})}}~\mathrm{,}
\end{displaymath}
where $L_{s+b}$ and $L_b$ are the maximum likelihood values for the fit to
the signal plus background and background only hypothesis, respectively.

A set of MC experiments is generated for a leptoquark mass and a value of 
the integrated luminosity.
The distribution of $S_L$ of the set of experiments is approximately gaussian 
and its median is used as the estimate of the significance.
The procedure is repeated for leptoquark masses of 250, 400 and 650~GeV and
different values of the integrated luminosity. 
 The results are shown in
Figure~\ref{fig:sign_vs_Lint_sysR}, which includes also the effect of the systematic 
uncertainties discussed in the next section. 

\subsection{Systematic uncertainties}

%The impact of systematic effects on the discovery potential and the determination 
%of the experiment sensitivity are not yet included in this study.

%A study for determining the sensitivity of the experiment will be performed.
%It will be used to set upper limits in case a signal is not observed. 
%It will be used an approach similar to the one used in \ref{highmassToMuons}.
%The same tools to generate MC experiments and perform log-likelihood fits will be used.

%Systematics: Luminosity(+_10\%), JES (+_10\%), ratio ttbar/Zgammajets, etc.

Once real data is available, and given the procedure used to extract the signal described in 
Section~\ref{sec:signalExtraction},
only the uncertainties that affect the knowledge of the shape of the distributions of 
the total background and the signal will have an impact on the significance, and have to be 
accounted for as systematic uncertainties.

The value of $R$, defined in Section~\ref{sec:signalExtraction}, 
directly affects the shape of the total background distribution used in the fit by changing the
mixture of the two backgrounds. 
The current knowledge, $R_{MC}$, of $R$ comes from the Monte Carlo
study and may introduce a deviation $\delta \equiv R/R_{MC}$ from the true value of $R$
due, for example, to imperfect knowledge of the relative magnitude of the two background 
cross sections and respective selection efficiencies.

$\tilde{R}$ is defined as the ratio between the number of $t\bar{t}$ and $Z/\gamma$+jets events in the
control samples described in Section~\ref{sec:bkgStudy}, and $\tilde{R}_{MC}$ is its
estimate from Monte Carlo. It is expected that the main reasons that
make $\delta$ deviate from unity affect $\tilde{R}/\tilde{R}_{MC}$ in the same way, assuming that the MC uncertainties are
dominated by the limited knowledge of the cross sections.
Therefore, once real data is available to calculate $\tilde{R}$ from the control samples of $t\bar{t}$ and $Z/\gamma$+jets, 
the estimate $\delta \simeq \tilde{R}/\tilde{R}_{MC}$ 
will be used to improve the knowledge of $R$ from the current value, $R_{MC}$, to $R_{data+MC} \equiv \delta \cdot R_{MC}$.

The uncertainty on the estimate of $\delta$ given by $\tilde{R}/\tilde{R}_{MC}$ will be 
dominated by the statistical error of $\tilde{R}$ from the number of events in the control samples, and will propagate to $R_{data+MC}$. 
To estimate the systematic effect on the significance, the fit is repeated by varying the 
value of $R$ by one standard deviation from its central value 
$R_{MC}$ \footnote{Not having real data yet, 
exact correctness of the MC is currently assumed by setting $\delta=1$, 
making effectively $R_{data+MC}=R_{MC}$.}, as shown in 
Figure~\ref{fig:sign_vs_Lint_sysR}.
%The systematic uncertainty due to $R$ is larger, for the same integrated luminosity, for a
%LQ mass of 250~GeV. This is due to the topology of the $M_{ej}$ distribution 
%of the signal that, at small LQ masses, peaks where the background does.
%For higher values of the LQ mass, the significance determination seems to be little 
%sensible to variations of $R$. This may be understood by considering the increased separation
%between the main features of the signal and backgound distributions, and the relative similarity 
%between the two backgound shapes: 
%the mixture between $t\bar{t}$ and $Z/\gamma$+jets events may vary, but
%the mixture between total background and signal does little.
As an example, for the point at $M_{LQ}=400$~GeV, integrated luminosity=100~$pb^{-1}$ and significance=12.7, 
the uncertainty on $R_{data+MC}$ is about 15\%, but this determines a change of 
significance of only 1\%.

The uncertainty 
of $R$ tends to be ``absorbed'' in the fitting procedure and have a limited impact on the 
significance determination.
In addition, this reassures that once the real data is available and the value of the correction 
factor $\delta$ can be estimated as described above,
even a significant departure from unity will not require a drastic re-assessment of
the current estimate of the luminosity needed for discovery.
The uncertainty on the knowledge of the signal shape used in the fit may also produce a systematic
uncertainty on the significance. A study of this effect will have to be performed.

In the case of setting an upper limit in the case of absence of signal, a larger number of systematic
uncertainties will have to be considered (on the integrated luminosity, the detector efficiency and resolution, the jet energy scale, and theoretical uncertainties) 
as they do have a direct impact on it. 
The procedure for setting an upper limit and the study of the 
systematics involved is not yet included in this version of the analysis.


 \begin{figure}
   \begin{center}
     \resizebox{13cm}{!}{\includegraphics{plots/sign_vs_Lint_sysR.eps}}
     \caption{Significance of the discovery as a function of the integrated luminosity.
       For each LQ mass, the fit to extract the significance is repeated varying the value of $R$
       (the ratio between the number of $t\bar{t}$ and $Z/\gamma$+jets events in the final sample)
       of one standard deviation from its central value. 
       The horizontal line represents a 5 sigma discovery.}
     \label{fig:sign_vs_Lint_sysR}
   \end{center}
 \end{figure}


\section{Conclusion}

A MC-based analysis of the pair production of first generation leptoquarks that decay to
an electron and a jet has been presented.
The analysis strategy and detector performance description corresponds to a 100$~pb^{-1}$ scenario.
Standard CMS techniques are used for electron and jet identification. 
A cut-based event selection has been applied and its optimization study will come in a future upgrade
of the analysis.
Several background sources have been studied and only two of them provide a significant contribution
after event selection. 
Data-driven techniques to understand the characteristics of these contributions have been developed 
and studied.

The signal component is extracted with a log-likelihood fit that does not rely on precise knowledge
of the level of the background but rather by exploiting the different shapes of the signal and background
distributions. 
The discovery potential of this analysis has been determined using a likelihood 
ratio estimator. This has shown that, for leptoquark masses just above the Tevatron exclusion limit of 290~GeV
\cite{d02008}, an early discovery is possible with a few $pb^{-1}$.
With an integrated luminosity of 100~$pb^{-1}$, discovery should be possible up
to LQ mass of about 500 GeV.
The impact of the systematics on the discovery potential has been discussed.
The procedure for setting an upper limit in absence of signal will be studied 
in the next upgrade of the analysis.


%--------------------------------------
%\clearpage
%--------------------------------------

\begin{thebibliography}{}

\bibitem{CMSTDR1}{The CMS Collaboration, Physics TDR volume I, The CMS Physics Technical Design Report  Volume 1},CERN-LHCC 2006-001 (2006)
  
\bibitem{CMSTDR2}{The CMS Collaboration, Physics TDR volume II, The CMS Physics Technical Design Report Volume 2}, CERN-LHCC 2006-002 (2006)
  
\bibitem {theories} {D.Acosta and S.K.Blessing, Ann.Rev.Nucl.Part.Sci 49,389},
  1999,
  {\em From June 2005}
\bibitem{hera}{The H1 collaboration, Search for leptoquark bosons in ep collisions at HERA}, hep-ex/0506044
  
\bibitem{d02008}{The D0 Collaboration, Search for First-Generation Leptoquarks in the dielectron channel with the D0 Detector in $p\bar{p}$ Collisions at $\sqrt{s}=1.96$ TeV}, Jun 2008,
  {\em D0 Note 5644-CONF}  %  {\em arXiv0710.0255v1}
  
\bibitem{Kramer}{M.Kramer et al., Pair production of scalar leptoquarks at the LHC},Jan 2008 ,{\em arXiv 0411038v2}
  %\cite{Abazov:2001mx}

\bibitem{PYTHIA}{Stephen Mrenna et al., PYTHIA web page}, http://home.thep.lu.se/~torbjorn/Pythia.html

\bibitem{GEANT4}{The GEANT4 Collaboration, GEANT 4 web page}, http://geant4.web.cern.ch/geant4/

\bibitem{Mangano:2002ea}
  M.~L.~Mangano, M.~Moretti, F.~Piccinini, R.~Pittau and A.~D.~Polosa,
  ``ALPGEN, a generator for hard multiparton processes in hadronic collisions,''
  JHEP {\bf 0307} (2003) 001
  [arXiv:hep-ph/0206293]
  %%CITATION = JHEPA,0307,001;%%
  

\bibitem{HLT_HE_VHE}{The CMS Collaboration, CMS High Level Trigger}, CERN/LHCC 2007-02 (2007) 
   	 

\bibitem{HEEP}{The CMS Collaboration, Search for high mass resonance production decaying into 
  an electron pair in the CMS experiment}, CMS PAS EXO-08-001
  
   
\bibitem{JetAlg} {The CMS Collaboration, Performance of Jet Algorithms in CMS}, CMS PAS JME-07-003
  
\bibitem{JES} {The CMS Collaboration, Plans for Jet Energy Corrections at CMS}, CMS PAS JME-07-002

\bibitem{Abazov:2001mx} 	 
  V.~M.~Abazov {\it et al.}  [D0 Collaboration], 	 
%  %``Search for first-generation scalar and vector leptoquarks,'' 	 
  Phys.\ Rev.\  D {\bf 64} (2001) 092004 	 
%  [arXiv:hep-ex/0105072]. 	 
%  %%CITATION = PHRVA,D64,092004;%%



  %\bibitem{HeepHlt}{D. Acosta et al., The CMS High Level Trigger}, CMS AN 2007/009,
  %\bibitem{GSFele}{S. Baffioni et al., Electron Reconstruction in CMS}, CMS NOTE 2006/040
  %\bibitem{EleID}{D. Newbold et al., Electron ID at High Energies}, CMS AN 2008/045,
  
  %  \bibitem{highmassToMuons}{\bf I. Altsybeev et al., Search for new high-mass resonances decaying to muon pairs in the CMS experiment}, CMS AN 2007/038,
  %\bibitem{JetAlg} {P. Schieferdecker et al., Performance of Jet Algorithms in CMS}, CMS AN-2008/01,
  
  %\bibitem{JES} {R. Harris, K. Kousouris, MC Truth L2 \& L3 Factorized Jet Corrections at CMS}, CMS AN-2008/03,

  
\end{thebibliography}

\end{linenumbers}
\end{document}
