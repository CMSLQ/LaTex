%\documentclass{cmspaper}
%\begin{document}

\section{Systematic Uncertainties} \label{sec:Systematics}

The main sources of systematic uncertainties for this analysis are discussed below.
The effects of these systematic uncertainties are taken into account in Section~\ref{CMSpotential} 
by summing them in quadrature when calculating discovery and exclusion potential, unless otherwise noted.
%
\begin{enumerate}
\item Uncertainty in the jet and electron energy scale

%Sarah's re-writes
To quantify the effect of the uncertainty on the reconstructed energy of the electron and jets,
the analysis is repeated rescaling the electron and jet energies
 with a factor. 
For selections optimized for different LQ masses, a change in the jet energy by +10\% (-10\%) leads 
to a change of +1 to +10\% (-3 to -15\%) in the signal efficiency and +33 to +52\% (-15 to -28\%) in the number of background events, 
%(from $2.29 \pm 0.58$ to $3.04 \pm 0.61$ when the jet energy is {\sl increased} 10\%)
while a +2\% (-2\%) variation in the electron energy scale yields a +1 to +2\% (-1 to -4\%) 
change in signal efficiency and a +1 to +9\% (-9 to -20\%) change in the number of background events. 

Jet and electron energy scale uncertainties
are not included for $t\bar{t}$ and $Z/\gamma$+jets backgrounds,
since they are determined using the data driven methods discussed
in Section~\ref{sec:bkgStudy}.

%In the early period of data taking the reconstructed energy of the electron and jets will not be known with 
%a high degree of certainty. To quantify the effect of this uncertainty on the number of signal and background events
%passing the selection criteria, the analysis is repeated with a multiplicative factor introduced to the energy 
%of jets and electrons within the event. 
%A variation of the jet energy of $\pm$10\% leads to a maximum change of approximately 
%7\% in the signal efficiency and 33\% in the number of background events, 
%%(from $2.29 \pm 0.58$ to $3.04 \pm 0.61$ when the jet energy is {\sl increased} 10\%)
%while a $\pm$5\% variation in the electron energy yields approximately a 3.5\% maximum 
%change in signal efficiency and 35\% in the number of background events. 
%%(from $2.29 \pm 0.58$ to $3.08 \pm 0.70$
%%when the electron energy is {\sl increased} 10\%).  
%%FIXME - rerun this for all mass points

\item Uncertainty in the integrated luminosity of the data

This uncertainty is estimated at 10\% for the first several months of LHC running~\cite{Ball:2007zza}. 
%
\item Statistical uncertainty on the MC data

The number of generated leptoquark signal events allows for a relatively small statistical uncertainty.
The uncertainty on the final number of selected events for the FullSim sample with a leptoquark mass of 400 GeV is 
approximately 0.4\%.  The number of events produced in the background samples, however, correspond to a much
smaller equivalent integrated luminosity.  
The statistical uncertainty in the number of MC events is summarized for signal and background samples 
in Table~\ref{tab:EventSelSummary} of Section~\ref{sec:eventSelection}.  
%
\item Uncertainties on FastSim selection efficiencies with respect to FullSim

The signal samples produced with FastSim show a slightly higher selection efficiency than FullSim, 
mostly due to the higher reconstruction efficiency of the electrons in the FastSim. 
This leads to a higher final selection efficiency in FastSim compared to FullSim by approximately 5\%, 
for leptoquark samples with a mass of 250 and 400 GeV. FullSim samples at higher mass are not available 
to perform the comparison. A conservative uncertainty of 10\% on the selection efficiency 
for FastSim samples is used in the whole mass range investigated. 
%
\item Uncertainty from the data driven background estimates

%Sarah's re-writes
Estimates of the background by data-driven techniques are affected by the statistical
uncertainties on the size of the control samples (that scales as the square root of the number of events).
The number of events expected in each control sample varies with the $S_T$ cut used, as 
described in section~\ref{sec:bkgStudy}. 
These uncertainties are calculated for an integrated luminosity of 100~pb$^{-1}$ 
and are used in Section~\ref{CMSpotential} only for the plots relative to this scenario.

The uncertainties on the ratios $R(p_{T})$ (see Equation~\ref{formula:NeejFromNemujj})
and $R_{OffZ/AtZ}$ (see Equation~\ref{formula:NeejjZ})
are expected to be small compared to the statistical uncertainty 
associated with the size of the control samples.

%Estimates of the background by data driven techniques are affected by the statistical
%uncertainties on the size of the control samples.
%The number of events expected in each control sample varies with the $S_T$ cut used, as 
%described in section~\ref{sec:bkgStudy}. These uncertainties can be correctly estimated 
%only when data is available, and they are not included in the discovery and exclusion potential
%for this analysis.

%Ellie + Francesco's corrections
\item Theoretical uncertainties 

The Parton Density Function (PDF) uncertainties are expected to give 
a major contribution to the theoretical uncertainties for this analysis.  
The effect of this systematic uncertainty has been estimated
using the method developed by the CTEQ collaboration~\cite{Martin:2003sk},
which is implemented in the CMS software as an event re-weighting technique~\cite{PDFRescaling}.
For this analysis we have used the PDF set CTEQ6.1.

%The CTEQ group fits a PDF function to a combination of data from current 
%and previous experiments, to extract the values and uncertainties of the 20 unknown parameters.
%Then the value of one of these 20 parameters is varied according to the its uncertainty, and the fit is performed again 
%to provide a new best fit function. This leads to 40 new possible PDFs corresponding to the positive or
%negative variation of each of the 20 parameters. All 41 (20 positive variations, 20 negative variations, and the best fit function) 
%functions are provided in a set by the CTEQ collaboration. For this analysis we have used the PDF set CTEQ6.1.

%The uncertainties associated with the PDF set is propagated to the analysis results 
%by applying a weight for each generated event, corresponding to each PDF variation. 
%The event weights are calculated using the momentum fraction carried by each incoming parton ($x_1, x_2$) 
%and the momentum scale ($Q$) of the event. The number of events passing the selection criteria for both 
%signal and background is calculated using the weights generated for each PDF variation. The largest variation 
%on the number of selected signal and background events is taken as an estimate of the systematic uncertainty. 

The uncertainty on the leptoquark cross section ranges 
from 5\% (for $LQ$ mass of 250 $GeV/c^2$) to 17\% (for $LQ$ mass of 600 $GeV/c^2$).  
The uncertainty on the signal efficiency is estimated to be less 
than 1\% in the whole mass range investigated.
%The uncertainties on the number of selected background events
%is approximately 8\% for $t\bar{t}$ events and 3\% for $Z/\gamma$+jets  
%events for the selection criteria corresponding to a $LQ$ mass of 400 $GeV/c^2$. 
A 10\% uncertainty on the number of selected background events is used for all the 
$LQ$ mass hypotheses. 

%FIXME - the ref PDFRescaling is only for the AN, not the PAS%
%FIXME - the ref PDFRescaling is only for the AN, not the PAS%
% reference https://twiki.cern.ch/twiki/bin/view/CMS/AachenPdfUncertainties.

%%%%%%%%%%%%%%%%%%%%%%%%
%Ellie 15 June 2009 
%%Theoretical uncertainties can affect the total cross section 
%%and the acceptance of the kinematic cuts, for both signal 
%%and background events. 
%At the LHC the PDF uncertainties are expected to contribute the most significant
%contribution to the theoretical uncertainties.  
%%At LHC the PDF uncertainties are expected to give the largest contribution. 
%The effect of this systematic uncertainty has been estimated
%using the method developed by the CTEQ collaboration~\cite{PDFRescaling,Martin:2003sk}.
%In calculating a PDF to be used by the event generation algorithm the CTEQ group fits
%the function to 20 parameters, the value and uncertainty of which are taken from the
%combination of data from current and previous experiments.
%If the value of any one of these 20 parameters is varied according to the uncertainty on that value 
%the best fit function will naturally change.  This leads to 40 new possible PDFs corresponding to the positive or
%negative variation of each of the 20 parameters.  All 41 functions are provided in a set by the CTEQ collaboration.
%%Each PDF is fit to known data using 20 free parameters, each with an associated uncertainty.  
%%Varying each parameter changes the best fit function for the PDF.  
%%This gives 40 new variations on each PDF.  

%%In order to avoid the large CPU time needed to regenerate events with 
%%a different PDF, 
%The uncertainty associated with the PDF is taken into account by calculating
%a weight for each event corresponding to each PDF variation 
%based on the momentum fraction carried by each incoming parton ($x_1, x_2$) 
%and the momentum scale of the event ($Q$).  
%%It is assumed the kinematics of each event does not change dramatically based 
%%on the PDF variation selected.  
%For this analysis we have used the PDF set CTEQ6.1.

%The number of events passing the selection criteria for both signal and background 
%is calculated using the weights generated for each PDF variation.  
%The number of signal events that pass chages by approximately 8\% for a leptoquark with a mass of 400 $GeV/c^2$, 
%while the change in the number of background events
%passing is approximately 8\% for $t\bar{t}$ events and 3\% for $Z/\gamma$+jet 
%events for the corresponding selection criteria.
%The contribution of other background processes is negligible.   
%The variation in the signal efficiency is about 0.2\% in the mass range considered.

%%FIXME - the ref PDFRescaling is only for the AN, not the PAS%
%%FIXME - the ref PDFRescaling is only for the AN, not the PAS%
%%[reference https://twiki.cern.ch/twiki/bin/view/CMS/AachenPdfUncertainties].
%The relative uncertainties on the differential cross sections 
%typically range from 2-5\% for Standard Model processes, 
%but may rise to 10\% or larger for parton processes at the TeV-scale 
%(like the leptoquark pair production).

\end{enumerate}
%
%\end{document}
