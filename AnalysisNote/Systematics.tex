%\documentclass{cmspaper}
%\begin{document}

\section{Systematic Uncertainties} \label{sec:Systematics}

Among the sources of systematic uncertainties on these results are uncertainties in the jet and electron energy, the integrated luminosity, statistical uncertainty on the simulated events, 
differences in efficiencies between the FastSim and FullSim, theoretical uncertainties and uncertainties associated with the data driven techniques to estimate 
background events.  Where possible the magnitude of these uncertainties is estimated.   Methods to use both data and further MC studies to better estimate these uncertainties are outlined here.
These effects are taken into account when calculating cross section limits and discovery potential unless otherwise noted.
%
\begin{enumerate}
\item Uncertainty in the jet and electron energy 

In the early days of data taking the reconstructed energy of the objects will not be known with a high degree of certainty.  To see the effect of this on the signal extraction the analysis 
is repeated with a multiplicative factor introduced to the jet and/or electron energy within the event.   A variation of the jet energy of 10\% leads to a change of approximately 7\% in the signal efficiency and a 33\% change in the number of background events (from $2.29 \pm 0.58$ to $3.04 \pm 0.61$ when the jet energy is {\sl increased} 10\%),
while a 5\% variation in the electron energy yields approximately a 3.5\% change in signal efficiency and a 35\% change in the number of background events (from $2.29 \pm 0.58$ to $3.08 \pm 0.70$ when the electron energy is {\sl increased} 10\%).  


\item Uncertainty in the integrated luminosity of the data

An additional uncertainty in the number of signal and background events comes from an uncertainty in the total integrated luminosity of the data.  
This is estimated at 10\% for the first several months of running, as is discussed in reference ... (put here reference to PTDR or other note with discussion of Lumi errors).

\item Statistical uncertainty on the MC data

The number of generated signal events allows for a relatively small statistical uncertainty.  The uncertainty on the final number of events for the FullSim sample with a leptoquark mass of 400 GeV is 
approximately 0.35\%.  The number of events produced in the background samples, however, correspond to a much smaller integrated luminosity.  The approximate error on the number of background events after the cuts
The statistical uncertainty in the number of MC events is summarized in table XXX in section YYY.  

\item Discrepancies between FullSim and FastSim efficiencies

The signal samples produced with FastSim show a slightly higher selection efficiency, largely due to the higher reconstruction efficiency of the 
electrons in the FastSim.  A conservative estimate on the difference between the FastSim and FullSim electron reconstruction efficiency is 10\%.  This leads to a higher final selection
 efficiency in FastSim compared to FullSim by approximately 2\% when directly comparing the leptoquark samples with a mass of 250GeV in both FastSim and FullSim.  
 
\item Theoretical uncertainties 

Theoretical uncertainties can affect both the total cross section and the acceptance of the kinematic cuts.  
The uncertainties due to variations in the PDF can be estimated by re-weighting the events based on the momentum fraction of each incoming parton.  
This method is described in [reference https://twiki.cern.ch/twiki/bin/view/CMS/AachenPdfUncertainties] and will be studied in future versions of this analysis.
%According to reference XXX  http://arxiv.org/pdf/0804.2800v3, page 6 the  NLO cross section for ttbar ranges between 396 and 455 pb due to variations in PDFs.  

\item Uncertainty from the data driven background estimates

Estimates of the background by data driven techniques have uncertainty that is dominated by the statistical error on the size of the control sample.
The number of events expected in each control sample varies with the $S_T$ cut used and is described in section XXX.  
In particular, the uncertainty on the ratio of events in the control sample to events in the signal sample for the method to estimate the 
Drell Yan background will benefit from the study using variations on the PDFs.
These estimates will be possible only once real data is available.

\end{enumerate}
%

%\end{document}
