\documentclass{cmspaper}
\begin{document}

\section{Systematic Uncertainties} \label{sec:Systematics}

Among the source of systematic uncertainties on these results are the following:
%
\begin{enumerate}
\item Uncertainty in the jet and electron energy 
\item Uncertainty in the integrated luminosity of the data
\item Statistical uncertainty on the MC data
\item Theoretical uncertainties on the cross section of both signal and background
\item Uncertainty from the data driven background estimates
\end{enumerate}
%
Where possible the magnitude of these uncertainties is estimated.   Methods to use both data and further MC studies to better estimate these uncertainties are outlined here.

In the early days of data taking the reconstructed energy of the objects will not be known with a high degree of certainty.  To see the effect of this on the signal extraction the analysis 
is repeated with a multiplicative factor introduced to the jet and/or electron energy within the event.  The results of this change in energy is shown in table \ref{tab:JetUncertainty} 
through table  \ref{tab:JetElecUncertainty}. 

\begin{table}[htbp]
\begin{center}
\begin{tabular}{|c|c|c|c|c|}
\hline
\hline
Jet Energy Mult. Factor & $N_{ev}$ $100pb^{-1}$ & \% diff & $N_{ev}$ $100pb^{-1}$ & \% diff  \\
\hline
\hline
0.90 & 37.687$\pm$ 0.137& 7\% &1.953$\pm$0.564& 15\% \\
\hline
0.95 & -$\pm$ -& -\% &-$\pm$-& -\% \\
\hline
1.05 & -$\pm$ -& -\% &-$\pm$-&-\% \\
\hline
1.10 & -$\pm$ -& -\% &-$\pm$-& -\% \\
\hline

\hline
\end{tabular}
\end{center}
\caption{Effect of jet energy scaling on number of signal and background events, rescaled for $100 pb^{-1}$, and the percentage difference with respect to unscaled energy.}
\label{tab:JetUncertainty}
\end{table}


\begin{table}[htbp]
\begin{center}
\begin{tabular}{|c|c|c|c|c|}
\hline
\hline
Electron Energy Mult. Factor & $N_{ev}$ $100pb^{-1}$ & \% diff & $N_{ev}$ $100pb^{-1}$ & \% diff  \\
\hline
\hline
0.95 & -$\pm$ -& -\% &-$\pm$-& -\% \\
\hline
0.98 & -$\pm$ -& -\% &-$\pm$-& -\% \\
\hline
1.02 & -$\pm$ -& -\% &-$\pm$-&-\% \\
\hline
1.05 & -$\pm$ -& -\% &-$\pm$-& -\% \\
\hline

\hline
\end{tabular}
\end{center}
\caption{Effect of electron energy scaling on number of signal and background events, rescaled for $100 pb^{-1}$, and the percentage difference with respect to unscaled energy.}
\label{tab:ElecUncertainty}
\end{table}

\begin{table}[htbp]
\begin{center}
\begin{tabular}{|c|c|c|c|c|c|}
\hline
\hline
Jet Energy Mult. Factor & Electron Energy Mult. Factor & $N_{ev}$ $100pb^{-1}$ & \% diff & $N_{ev}$ $100pb^{-1}$ & \% diff  \\
\hline
\hline
0.90 & 0.95 & -$\pm$ -& -\% &-$\pm$-& -\% \\
\hline
1.10 & 1.05 & -$\pm$ -& -\% &-$\pm$-& -\% \\
\hline

\hline
\end{tabular}
\end{center}
\caption{Effect of combined jet and electron energy scaling on number of signal and background events, rescaled for $100 pb^{-1}$, and the percentage difference with respect to unscaled energy.}
\label{tab:JetElecUncertainty}
\end{table}


An additional uncertainty in the number of signal and background events comes from an uncertainty in the total integrated luminosity of the data.  
This is estimated at 10\% for the first several months of running.
The contribution of these uncertainties to the limit on the signal cross section and discovery potential is discussed in section XXX.

The statistical uncertainty in the number of MC events is summarized in table XXX in section YYY.  This uncertainty is taken into account when calculating the upper 
limit on the cross section for the optimization of the ST cut.

The signal samples produced with the fast simulation show a slightly higher selection efficiency, largely due to the higher reconstruction efficiency of the 
electrons in the fast simulation.  A conservative estimate on the difference between the fast simulation and full simulation selection efficiency is 10\%.  This uncertainty 
applies specifically to the exclusion and discovery limits as these are produced using the fast simulation signal samples.

Theoretical uncertainties in the cross section of both signal and background will be studied in future versions of this analysis.  The uncertainties due to variations in the 
PDF can be estimated by re-weighting the events based on the momentum fraction of each incoming parton.  According to reference XXX 
%http://arxiv.org/pdf/0804.2800v3, page 6 
the  NLO cross section for ttbar ranges between 396 and 455 pb due to variations in PDFs.  

Estimates of the background by data driven techniques have uncertainty that is dominated by the statistical error on the size of the control sample, 
as described in section XXX.  Once real data is available the uncertainty can be better estimated based on the agreement with MC and ...

\end{document}
