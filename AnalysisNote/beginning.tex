
%\documentclass{cmspaper}
%\begin{document}

\begin{titlepage}

% select one of the following and type in the proper number:
   %\cmsnote{2008/000}
  \cmsan{2008/000}
%  \internalnote{2005/000}
%  \conferencereport{2005/000}
   \date{30th October 2008}


  \title{Search for First Generation Scalar Leptoquarks at the CMS experiment}

  \begin{Authlist}
   Sarah Eno, Paolo Rumerio, Francesco Santanastasio, Elizabeth Twedt
       \Instfoot{cern}{University of Maryland, College Park, MD, USA}
  \end{Authlist}


  \begin{abstract}
    The discovery potential for pair production of first generation scalar leptoquarks 
    decaying to an electron and quark is investigated 
    at the CMS experiment using Monte Carlo samples produced with a full simulation of the detector response.  
    The analysis strategy is defined assuming an available integrated luminosity of 100 pb$^{-1}$.
    Event selection and reconstruction and identification of high energy electrons and jets are discussed.
    The electron-jet invariant mass distribution is reconstructed
    and used to identify the possible presence of a leptoquark signal.
    Data-driven techiques are discussed to determine the main Standard Model backgrounds using 
    control samples. The signal yield is extracted using a maximum likelihood fit to the 
    electron-jet invariant mass distribution, and the significance of the measurement is estimated.
    Finally the CMS discovery potential is quantified for different leptoquark mass hyphotheses.

%The leptoquarks are assumed to decay exclusively to electrons and quarks.  Detailed explanations of the discovery techniques and data driven methods to extract the signal from background events are given.  
%Both exclusion limits and discovery potential are discussed for masses of the leptoquark ranging from around the current limit of 251 GeV to 1 TeV.
  \end{abstract} 

\end{titlepage}

\setcounter{page}{2}%JPP

\section{Introduction}

This note describes the analysis techniques that will be used to search for evidence of first generation scalar leptoquarks 
at the CMS experiment. The methods described are evaluated using Monte Carlo (MC) simulations assuming an availability of 
100 pb$^{-1}$ of integrated luminosity and 7 TeV proton beams.

Leptoquarks (LQ) are conjectured exotic particles with both baryon and lepton number, allowing their decay into a lepton and quark.    
Several models of new physics beyond the Standard Model (SM) predict the existence of leptoquarks, including General Unified Theories, 
technicolor and composite models \cite{theories}.  
%% OTHER MODELS (SUSY WITH R-PARITY???)

These exotic particles carry color and fractional electric 
charge, and may be either scalar or vector particles. Three generations of leptoquarks are predicted, 
corresponding to the generations of quarks and leptons in the SM.  Intergenerational coupling of leptoquarks is forbidden 
due to the experimental limits on flavor changing neutral currents. 

The model parameters are 1) $M_{LQ}$, the LQ mass, 2) $\beta$, the branching ratio of the decay 
$\mbox{LQ} \rightarrow l + q$, where $l$ is a charged lepton and $q$ is a quark, 
3) $\lambda_{\mbox{LQ}lq}$, the coupling between LQ, lepton and quark ($\mbox{LQ}-l-q$ vertex). 
The complementary decay with a branching ratio of 1-$\beta$ describes the decay $\mbox{LQ} \rightarrow \nu + q$, 
where $\nu$ is a nutrino.
The results of the HERA experiments show that the data are in agreement with a scenario of weak 
$\lambda_{\mbox{LQ}lq}$ coupling \cite{hera}. The choice is to fix the paramameter 
$\lambda_{\mbox{LQ}lq} = \lambda_{EM} \approx 0.3$, which gives 
a theoretical LQ relative width ($\Gamma_{LQ}/M_{LQ}$) around 0.2\%. 
Therefore the measurement of $\Gamma_{LQ}/M_{LQ}$ cannot be performed at LHC, since 
it is completely dominated by the detector resolution.  

Experiments at the Tevatron have placed the current lower limit on the mass of first generation scalar 
leptoquarks around 290 GeV (assuming $\beta=1$ in the electron channel) \cite{d02008}.
At the LHC, the dominant production modes of leptoquarks will be through gluon-gluon fusion and 
quark-antiquark annihilation producing a leptoquark-antileptoquark pair. 
The cross section of these two processes is almost independent from the value of 
$\lambda_{\mbox{LQ}lq}$, since there is no $\mbox{LQ}-l-q$ vertex in the Feynman diagram 
at leading order. This note studies the signal of first generation scalar leptoquarks decaying into electron and 
(light) quark with a branching ratio $\beta=1$. 
For this scenario, the experimental signature of a LQ event is striking: two 
high transverse momentum electrons and two high transverse momentum jets 
whose combination gives a peak in the electron-jet invariant mass 
spectrum corresponding to the LQ mass. No true missing transverse energy is expected in LQ events.

This note is organized as follows. 
The signal and background MC samples used in this study are described in section \ref{sec:MCSamples}.
The online selection and the trigger efficiency to select LQ events are discussed in section \ref{sec:trig}.
The LQ signal is reconstructed using mainly data from the
electromagnetic (ECAL) and hadronic (HCAL) calorimeters for energy measurements, while tracker information 
is mostly used for electron identification and isolation. 
The reconstruction and identification of high energy electrons and jets are described respectively 
in section \ref{sec:electrons} and \ref{sec:jet}.
Techniques used here to identify the LQ signal include a search for features in the invariant mass
spectrum of the electron-jet pairs, and data driven techniques to estimate 
and separate the dominant background processes.
In section \ref{sec:eventSelection}, the analysis strategy and the efficiencies of the selection criteria 
are discussed. Data-driven techniques for background estimate are presented in section \ref{sec:bkgStudy}.
Finally the procedure for signal extraction, and the CMS discovery potential in this channel are described
respectively in section \ref{sec:signalExtraction} and \ref{CMSpotential}.


%Experiments at the Tevatron have placed the current lower mass limit of the first generation scalar leptoquark at 251 GeV (assuming a branching ratio of 1 into electrons) \cite{d02007}.  This study includes 
%signal samples produced at a range of masses from 250 GeV to 1 TeV.  Monte Carlo simulation studies show the reach of CMS to be leptoquarks of masses around 1 TeV.  Assuming a mass of the leptoquark at the 
%current limit the studies show a $5 \sigma$ discovery possibility with just above $5pb^{-1}$ of integrated luminosity.  With the full statistics used in this study of  $100pb^{-1}$  a leptoquark of mass close 
%to 500 GeV could be discovered with a significance of $5 \sigma$.
%Limits if no discovery?

\section{Monte Carlo Samples} \label{sec:MCSamples}

%By far the dominant process is that of gluon-gluon 
%fusion, as can been seen in table~\ref{tab:CTEQ6} where the leading order and next to leading order cross sections are shown, 
%as well as the ratio of the processes contributing to the total cross section.  
%A more detailed description of the calculations of these numbers can be found in ~\cite{Kramer}.

%  \begin{table}[htb]
%    \caption{\small \sl Cross Section and K-factors using CTEQ6L1 and CTEQ6M parton densities for LQ productions at a variety of LQ masses\cite{Kramer} }
%    \label{tab:CTEQ6}
%    \begin{center}
%      \begin{tabular}{|l|cccc|} \hline
%           $M_{LQ}$ (GeV) & $\sigma_{LO}$ & $\sigma_{NLO}$ & gg:qq:gq & K \\ \hline
%        200  & $0.500\times10^{2}$ &  $0.742\times10^{2}$ &  $0.94:0.05:0.01$ &  $1.48$  \\ \hline
%        400  & $0.140\times10^{1}$ &  $0.224\times10^{1}$ &  $0.91:0.10:-0.01$ &  $1.60$  \\ \hline
%        600  & $0.135$ &  $0.225$ &  $0.88:0.15:-0.03$ &  $1.67$  \\ \hline
%        800  & $0.219\times10^{-1}$ &  $0.378\times10^{-1}$ &  $0.84:0.19:-0.03$ &  $1.73$  \\ \hline
%        1000  & $0.471\times10^{-2}$ &  $0.836\times10^{-2}$ &  $0.82:0.22:-0.04$ &  $1.77$  \\ \hline
%      \end{tabular}
%    \end{center}
%  \end{table}

MC signal samples are generated with leptoquark masses ranging from 250 GeV to 1 TeV and $\beta=1$. 
At the LHC the three main processes by which leptoquark pair production could occur are gluon-gluon fusion, 
quark-antiquark annihilation, and quark-gluon fusion. In the mass range investigated the dominant process is gluon-gluon fusion. 

The Full Simulation (FullSim) signal samples are produced using 
CMSSW\_1\_4\_12 (pythia version 6.227 and GEANT 4) 
for generation and simulation, and CMSSW\_1\_6\_7  for digitization and reconstruction. 
These CMSSW versions are chosen in order to produce samples
compatible with the officially produced background samples described later in this section. 
These signal samples have been produced with a 100 $\mbox{pb}^{-1}$ mis-calibration and mis-alignment 
scenario of the detector performances. Table~1
%\ref{tab:NumEvents} 
shows the number of events and the cross sections at leading order (LO) for samples generated at different LQ mass.

Additionally, samples produced with the Fast Simulation (FastSim) in CMSSW version 1\_6\_9 are 
studied in order to validate the quality of the FastSim process. The FastSim uses a parameterization 
of the detector response instead of using the full GEANT-based simulation to determine the detector response.
The FastSim samples are created with the same mis-calibration and mis-alignment scheme as the FullSim samples. 
%Table ~\ref{tab:FastVsFullSim} shows the relative time and size advantage of using the FastSim over the FullSim for leptoquark event production.  
The use of the FastSim would be highly beneficial to this analysis, as the production of events with such large EM showers, and large numbers of particles generated 
are quite time consuming in the FullSim.  
%The comparison of the MC events generated with FullSim to the FastSim shows an agreement at few \% level in 
%the shape of reconstructed quantities, and some discrepancies around 5-10\% in the reconstruction efficiencies. 
Plots comparing various reconstructed quantities are included in some sub-sections of this note.  
The main results of this analysis, including selection efficiencies and discovery potential are obtained with FullSim samples.

%\begin{table}[htb]
%  \label{tab:NumEvents}
%  \begin{center}
%    \begin{tabular}{|l|ccc|} \hline
%%      & \multicolumn{4}{c|}{Leptoquark Mass (GeV)} \\ 
%      LQ mass (GeV) & FullSim & FastSim & $\sigma_{LO}$ (pb) \\ \hline
%      250  & 1k  &  20k  &  14   \\
%      400  & 1k  &  20k  & 1.2    \\
%      650  & 2.5k &  20k  & 0.076    \\
%      1000 & 1k  &  20k  & 0.0047    \\
%      \hline
%    \end{tabular}
%    \caption{\small \sl Number of generated events and cross sections at LO, for LQ MC samples with various masses.  
%      FullSim samples made with CMSSW\_1\_4\_12 + CMSSW\_1\_6\_7, FastSim samples made with CMSSW\_1\_6\_9.}
%  \end{center}
%\end{table}
%
\begin{table}[htb]
  \label{tab:NumEvents}
  \begin{center}
    \begin{tabular}{|l|ccc|} \hline
%      & \multicolumn{4}{c|}{Leptoquark Mass (GeV)} \\ 
      LQ mass (GeV) & FullSim & FastSim & $\sigma_{LO}$ (pb) \\ \hline
      250  & 1k  &  20k & 14 \\
      400  & 1k  &  20k & 1.2 \\
      650  & 2.5k &  20k & 0.076 \\
      1000 & 1k  &  20k & 0.0047 \\
      \hline
    \end{tabular}
    \caption{\small \sl Number of generated events and cross sections at LO, for LQ MC samples with various masses.  
      FullSim samples made with CMSSW\_1\_4\_12 + CMSSW\_1\_6\_7, FastSim samples made with CMSSW\_1\_6\_9.}
  \end{center}
\end{table}

 
%from : twiki.cern.ch/twiki/bin/view/CMS/CSA07Physics
%https://twiki.cern.ch/twiki/bin/view/CMS/GeneratorProduction2007CSA07


%  \begin{table}[htb]
%    \caption{\small \sl Average time needed to produce one event and size of one event for samples in FastSim and FullSim}
%    \label{tab:FastVsFullSim}
%    \begin{center}
%      \begin{tabular}{|l|c|cccc|} \hline
%               & \multicolumn{1}{c|}{FullSim} & \multicolumn{4}{c|}{FastSim} \\ 
%        Leptoquark Mass (GeV) &  650 & 250 & 400 & 650 & 1000  \\ \hline
%        Time per Event (sec) & 330  &  X  & 2.1  & 2.4  &  2.7     \\
%        Size of Event (kbytes)  & 3.5M (Reco) & Y & 116k &  123k & 132k  \\ \hline
%
%      \end{tabular}
%    \end{center}
%  \end{table}

%twiki.cern.ch/twiki/bin/view/CMS/FastSimPhysicsValidationInfo
%twiki.cern.ch/twiki/bin/view/CMS/SWGuideFastSimulation

The SM background samples are taken from the CSA07 official production,
which consists of more than 150M events of various SM processes roughly representing the first 100 pb$^{-1}$ of LHC data.
All events are generated and simulated with CMSSW\_1\_4\_12 and reconstructed with CMSSW\_1\_6\_7. 
The CSA07 samples used for this analysis are listed below:
\begin{itemize}
%
\item $t\bar{t}$+Njet events (where N=0,1,..,4), generated with ALPGEN \cite{Mangano:2002ea} inclusive production (all decays), 
LO-NLO K factor of 1.85 applied 
in the cross section;\footnote{The K factor of 1.85 applied to ttbar events is a default in the software provided by CMS to 
use these data samples.  It comes from a MCFM NLO vs LO study.}
%
\item $Z/\gamma$+Njet events (where N=0,1,..,5), generated with ALPGEN, $ 0 < P_{T}^{Z/\gamma} < 300 $ GeV and $40<M_{Z/\gamma}<200$ GeV, 
$Z/\gamma*$ decaying into charged leptons;  
%
\item $W$+Njet events (where N=0,1,..,5), generated with ALPGEN, $ 0 < P_{T}^{W} < 300 $ GeV, $W$ decaying into leptons;  
%
\item QCD multi-jet events, generated with PYTHIA, inclusive production in $P_{T}^{jet}$ bins from 0 to $\infty$;  
%
\item $\gamma$+jet events, generated with PYTHIA, $ 0 < P_{T}^{jet} < 7000 $ GeV;  
%
\item $WW$, $WZ$, $ZZ$ events, generated with PYTHIA, inclusive production (all decays).
\end{itemize} 

%In the CSA07 production there were three flavors of Standard Model event combinations, two of which were used in this analysis: 
%the {\sl Chowder} contains ALPGEN W+jet, Z+jet and tt+jet samples, and the {\sl Gumbo} which contains QCD, Photon+jets, MinBias.  
%The event samples for each process within the soups were produced with numbers of events corresponding to a variety of integrated luminosities. 
%In order to scale the sample sizes correctly a central module that calculates the event weights has been used ({\sl CSA07EventWeightProducer.cc}) . 

%\end{document}
