
%\documentclass{cmspaper}
%\begin{document}
\begin{titlepage}

% select one of the following and type in the proper number:
   %\cmsnote{2008/000}
  \analysisnote{2008/070}
%  \internalnote{2005/000}
%  \conferencereport{2005/000}
   \date{27 May 2009}


  \title{Search for Pair Production of First Generation Scalar Leptoquarks at the CMS Experiment}

  \begin{Authlist}
   Sarah Eno, Dinko Ferencek, Paolo Rumerio, Francesco Santanastasio, Elizabeth Twedt
       \Instfoot{cern}{University of Maryland, College Park, MD, USA}
  \end{Authlist}


  \begin{abstract}
%Sarah's re-write
    We investigate the discovery potential of the CMS experiment
    for pair production of first generation scalar leptoquarks that 
    decay to an electron and quark, assuming 
    an integrated luminosity of 100 pb$^{-1}$ and $pp$ collisions 
    at $\sqrt{s}=10$ TeV.
    We discuss 
    reconstruction and identification of high-energy electrons and jets, 
    and optimization of the event selection.
    Data-driven techiques are used to determine 
the main standard model backgrounds. 
The CMS discovery and exclusion potentials for different leptoquark 
mass hyphotheses are presented.



%    The discovery potential for pair production of first generation scalar leptoquarks that 
%    decay to an electron and quark is investigated 
%    at the CMS experiment using Monte Carlo samples produced with a full simulation of detector response.  
%    The analysis strategy assumes an integrated luminosity of 100 pb$^{-1}$ and $pp$ collisions 
%    at $\sqrt{s}=10$ TeV.
%    Reconstruction and identification of high energy electrons and jets, 
%    and optimized selection of events are discussed.
%    The electron-jet invariant mass distribution is reconstructed
%    and used to identify the possible presence of a leptoquark signal.
%    Data-driven techniques are discussed to determine the main standard model backgrounds through use of 
%    control samples. The CMS discovery and exclusion potentials using a counting experiment approach 
%    are quantified for different leptoquark mass hypotheses.
%
%The leptoquarks are assumed to decay exclusively to electrons and quarks.  Detailed explanations of the discovery techniques and data driven methods to extract the signal from background events are given.  
%Both exclusion limits and discovery potential are discussed for masses of the leptoquark ranging from around the current limit of 251 GeV to 1 TeV.
  \end{abstract} 

\end{titlepage}

\setcounter{page}{2}%JPP

\section{Introduction}
%Sarah's re-write
This note describes analysis techniques 
appropriate for the
search for first generation scalar leptoquarks 
using data from the CMS experiment taken with 5 TeV proton beams and 
corresponding to 100 pb$^{-1}$ of integrated luminosity.
The methods described are evaluated using 
Monte Carlo (MC) simulations.

%This note describes the analysis techniques that will be used to search for evidence of 
%first generation scalar leptoquarks 
%at the CMS experiment. 
%The methods described are evaluated using Monte Carlo (MC) simulations assuming a start-up scenario of LHC
%with about 100 pb$^{-1}$ of integrated luminosity and 5 TeV proton beams.
%
Leptoquarks (LQ) are new exotic particles conjectured to have both baryon and lepton number 
and primary decays into leptons and quarks. 
Several models of physics beyond the standard model (SM) predict the existence of leptoquarks,
including General Unified Theories, Technicolor, and composite models \cite{Acosta:1999ws}.  
%% OTHER MODELS (SUSY WITH R-PARITY???)

Leptoquarks carry color and fractional electric charge, 
and can be either scalar or vector particles. The three generations of predicted leptoquarks 
correspond to the families of quarks and leptons in the SM.  Intergenerational coupling of 
leptoquarks is highly constrained by the experimental limits on flavor-changing neutral currents
and proton decay~\cite{Acosta:1999ws,Davidson:1993qk}. 
%FIXME% \cite{ADDREFERENCE}. 


%Sarah's re-write
The parameters of the model are i) $M_{LQ}$, the LQ mass, ii) $\beta$, 
the branching fraction 
$\mbox{LQ} \rightarrow l + q$
where $l$ is a charged lepton and $q$ is a quark, and
iii) $\lambda_{\mbox{LQ}lq}$, the coupling between the LQ, lepton and 
quark ($\mbox{LQ}-l-q$ vertex). 
The complementary decay $\mbox{LQ} \rightarrow \nu + q$, 
where $\nu$ is a weak-isospin partner of the lepton,
has branching fraction 1-$\beta$.
Results from the experiments at the HERA accelerator
give upper limits on the LQ production cross section that restrict  
$\lambda_{\mbox{LQ}lq}$ to be small (comparable or less than the strength of 
the electromagnetic coupling $\lambda_{EM}$) for $M_{LQ}<300$ GeV~\cite{Aktas:2005pr}. 
In this analysis we assume the parameter 
$\lambda_{\mbox{LQ}lq} = \lambda_{EM} \approx 0.3$, 
that leads to $\Gamma_{LQ}/M_{LQ}$ of about 0.2\% for scalar leptoquarks. 
%Consequently, 
%$\Gamma_{LQ}/M_{LQ}$ can not be measured at the LHC, 
%since the width the LQ mass distribution will be 
%dominated by detector resolution.  


%The parameters of the model are i) $M_{LQ}$, the LQ mass, ii) $\beta$, the branching fraction for the decay 
%$\mbox{LQ} \rightarrow l + q$, where $l$ is a charged lepton and $q$ is a quark, and
%iii) $\lambda_{\mbox{LQ}lq}$, the coupling between LQ, lepton and quark ($\mbox{LQ}-l-q$ vertex). 
%The complementary decay of a branching ratio 1-$\beta$ describes the decay $\mbox{LQ} \rightarrow \nu + q$, 
%where $\nu$ is a weak-isospin partner of the lepton.
%Results from HERA on upper limits for LQ production restrict  
%$\lambda_{\mbox{LQ}lq}$ to relatively weak coupling\cite{Aktas:2005pr}. For example, fixing the parameter 
%$\lambda_{\mbox{LQ}lq} = \lambda_{EM} \approx 0.3$, leads to 
%a theoretical relative width for the LQ of ($\Gamma_{LQ}/M_{LQ}$) around 0.2\%. 
%Consequently, $\Gamma_{LQ}/M_{LQ}$ will not be able to be measured at the LHC, since 
%the width of any peak on the mass distribution of the LQ will be dominated by detector resolution.  
%

%Sarah's re-write
Experiments at the Tevatron have placed the most stringent lower 
limit of approximately 290 $GeV/c^2$ (assuming $\beta=1$ for the electron channel) 
on the mass of first-generation scalar leptoquarks~\cite{d02008},\cite{Acosta:2005ge}.
At the LHC, pair production of leptoquarks takes place 
mainly through gluon-gluon fusion and 
quark-antiquark annihilation (dominated by the former). 
The cross sections for both these sub-processes are almost 
independent of the value of 
$\lambda_{\mbox{LQ}lq}$, since there is 
no $\mbox{LQ}-l-q$ vertex in the Feynman diagram for LQ pair production 
at leading order. 
The production of a single leptoquark in association with a lepton 
is also possible via quark-gluon 
fusion at a lower rate. 
The cross section for single leptoquark production
becomes comparable to the one for pair production only 
for $M_{LQ}\approx 1$ TeV \cite{Belyaev:2005ew}, 
which is well above the experimental reach of this start-up analysis.  
Cross sections for single and pair production of vector leptoquarks 
are in general expected to be larger than scalar leptoquarks \cite{Belyaev:2005ew}.

%Experiments at the Tevatron have placed the most stringent lower limit on the mass of first-generation scalar 
%leptoquarks of approximately 290 GeV (assuming $\beta=1$ for the electron channel) \cite{d02008},\cite{Acosta:2005ge}.
%At the LHC, the pair production of leptoquarks will take place mainly through gluon-gluon fusion and 
%quark-antiquark annihilation, yielding $LQ + \overline{LQ}$. 
%The cross sections for the two sub-processes are almost independent on the value of 
%$\lambda_{\mbox{LQ}lq}$, since there is no $\mbox{LQ}-l-q$ vertex in the Feynman diagram for LQ pair production 
%at leading order. At lower rate, the production of single leptoquark in association with a lepton is also possible via quark-gluon 
%fusion, yielding $LQ+ l$ or $LQ+ \nu$ (and relative final states with antiparticles). The cross section of single leptoquark production
%becomes comparable to the one of pair production only for $M_{LQ}\approx 1$ TeV \cite{Belyaev:2005ew}, 
%which is well above the experimental reach of this start-up analysis (focused on the first 100 pb$^{-1}$ of data at $\sqrt{s}=10$ TeV).  
%

This note examines the potential reach of a search for
pair production of first-generation
scalar leptoquarks that decay into electrons and
(light) quarks with an unknown branching fraction $\beta$.
This process yields an experimental signature that
is quite striking, with two
high transverse momentum ($p_T$) electrons
and two high $p_T$ jets (eejj channel), and
a peak in the electron-jet invariant mass
spectrum that corresponds to the LQ mass.
No true missing transverse momentum is expected.


%This note examines the signal for pair production of first-generation scalar leptoquarks that decay into electrons and 
%(light) quarks with a branching ratio $\beta=1$. 
%For this scenario, the experimental signature of a LQ event is quite striking with two 
%high transverse momentum ($p_T$) electrons and two high $p_T$ jets (eejj channel), 
%whose combination yields a peak in the electron-jet invariant mass 
%spectrum that corresponds to the LQ mass. No true missing transverse momentum is expected in LQ events.
%
This note is organized as follows. 
The signal and background MC samples used in this study are described in Section \ref{sec:MCSamples}.
The online requirements and the trigger efficiency for selecting LQ events are discussed in Section \ref{sec:trig}.
The LQ signal is reconstructed using mainly energy measurements from the
electromagnetic (ECAL) and hadronic (HCAL) calorimeters, while tracker information 
is used primarily to identify and select electrons. 
The reconstruction and identification of high energy electrons and jets are described, respectively, 
in Sections \ref{sec:electrons} and \ref{sec:jet}.
Techniques used to identify the LQ signal include optimization of selection criteria based on 
reconstructed quantities to discriminate between signal and background, 
and data-driven techniques to estimate the dominant background processes.
In Section \ref{sec:eventSelection}, the analysis strategy, the optimization of the selection criteria, 
and the relative selection efficiencies are discussed. 
Data-driven techniques for estimating background are presented in Section \ref{sec:bkgStudy}.
Finally, the estimate of the main systematic uncertainties, and the CMS discovery/exclusion potential 
in the eejj channel are described, respectively, in Section \ref{sec:Systematics} and \ref{CMSpotential}.


%Experiments at the Tevatron have placed the current lower mass limit of the first generation scalar leptoquark at 251 GeV (assuming a branching ratio of 1 into electrons) \cite{d02007}.  This study includes 
%signal samples produced at a range of masses from 250 GeV to 1 TeV.  Monte Carlo simulation studies show the reach of CMS to be leptoquarks of masses around 1 TeV.  Assuming a mass of the leptoquark at the 
%current limit the studies show a $5 \sigma$ discovery possibility with just above $5pb^{-1}$ of integrated luminosity.  With the full statistics used in this study of  $100pb^{-1}$  a leptoquark of mass close 
%to 500 GeV could be discovered with a significance of $5 \sigma$.
%Limits if no discovery?


%\end{document}
