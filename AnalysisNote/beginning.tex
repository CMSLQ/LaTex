
%\documentclass{cmspaper}
%\begin{document}

\begin{titlepage}

% select one of the following and type in the proper number:
   %\cmsnote{2008/000}
  \cmsan{2008/000}
%  \internalnote{2005/000}
%  \conferencereport{2005/000}
   \date{12 May 2009}


  \title{Search for First Generation Scalar Leptoquarks at the CMS Experiment}

  \begin{Authlist}
   Sarah Eno, Paolo Rumerio, Francesco Santanastasio, Elizabeth Twedt
       \Instfoot{cern}{University of Maryland, College Park, MD, USA}
  \end{Authlist}


  \begin{abstract}
    The discovery potential for pair production of first generation scalar leptoquarks that 
    decay to an electron and quark is investigated 
    at the CMS experiment using Monte Carlo samples produced with a full simulation of detector response.  
    The analysis strategy assumes an integrated luminosity of 100 pb$^{-1}$.
    Event selection and reconstruction and identification of high energy electrons and jets are discussed.
    The electron-jet invariant mass distribution is reconstructed
    and used to identify the possible presence of a leptoquark signal.
    Data-driven techiques are discussed to determine the main standard model backgrounds through use of 
    control samples. The signal and its significance are extracted using XXXXXX.
    Finally, the CMS discovery potential is quantified for different leptoquark mass hyphotheses.

%The leptoquarks are assumed to decay exclusively to electrons and quarks.  Detailed explanations of the discovery techniques and data driven methods to extract the signal from background events are given.  
%Both exclusion limits and discovery potential are discussed for masses of the leptoquark ranging from around the current limit of 251 GeV to 1 TeV.
  \end{abstract} 

\end{titlepage}

\setcounter{page}{2}%JPP

\section{Introduction}

This note describes the analysis techniques that will be used to search for evidence of first generation scalar leptoquarks 
at the CMS experiment. The methods described are evaluated using Monte Carlo (MC) simulations assuming 
100 pb$^{-1}$ of integrated luminosity and 5 TeV proton beams.

Leptoquarks (LQ) are new exotic particles conjectured to have both baryon and lepton number and primary decays into leptons and quarks.    
Several models of physics beyond the standard model (SM) predict the existence of leptoquarks, including General Unified Theories, 
technicolor and composite models \cite{theories}.  
%% OTHER MODELS (SUSY WITH R-PARITY???)

Leptoquarks carry color and fractional electric 
charge, and can be either scalar or vector particles. The three generations of predicted leptoquarks 
correspond to the families of quarks and leptons in the SM.  Intergenerational coupling of leptoquarks is forbidden 
a priori by the experimental upper limits on flavor-changing neutral currents. 

The parameters of the model are i) $M_{LQ}$, the LQ mass, ii) $\beta$, the branching fraction for the decay 
$\mbox{LQ} \rightarrow l + q$, where $l$ is a charged lepton and $q$ is a quark, and
iii) $\lambda_{\mbox{LQ}lq}$, the coupling between LQ, lepton and quark ($\mbox{LQ}-l-q$ vertex). 
The complementary decay of a branching ratio 1-$\beta$ describes the decay $\mbox{LQ} \rightarrow \nu + q$, 
where $\nu$ is a weak-isospin partner of the lepton.
Results from HERA on upper limits for LQ production restrict  
$\lambda_{\mbox{LQ}lq}$ to relatively weak coupling\cite{hera}. For example, fixing the paramameter 
$\lambda_{\mbox{LQ}lq} = \lambda_{EM} \approx 0.3$, leads to 
a theoretical relative width for the LQ of ($\Gamma_{LQ}/M_{LQ}$) around 0.2\%. 
Consequently, $\Gamma_{LQ}/M_{LQ}$ will not be able to be measured at the LHC, since 
the width of any peak on the mass distribution of the LQ will be dominated by detector resolution.  

Experiments at the Tevatron have placed the most stringent lower limit on the mass of first-generation scalar 
leptoquarks of approximately 290 GeV (assuming $\beta=1$ for the electron channel) \cite{d02008}.
At the LHC, the production of leptoquarks will take place mainly through gluon-gluon fusion and 
quark-antiquark annihilation, yielding $LQ + \bar(LQ)$. 
The cross sections for the two sub-processes are almost independent of the value of 
$\lambda_{\mbox{LQ}lq}$, since there is no $\mbox{LQ}-l-q$ vertex in the Feynman diagram for LQ pair production 
at leading order. This note examines the signal for first-generation scalar leptoquarks that decay into electrons and 
(light) quarks with a branching ratio $\beta=1$. 
For this scenario, the experimental signature of a LQ event is quite striking with two 
high transverse momentum ($p_T$) electrons and two high $p_T$ jets, 
whose combination yields a peak in the electron-jet invariant mass 
spectrum that corresponds to the LQ mass. No true missing transverse momentum is expected in LQ events.

This note is organized as follows. 
The signal and background MC samples used in this study are described in Section \ref{sec:MCSamples}.
The online requirements and the trigger efficiency for selecting LQ events are discussed in Section \ref{sec:trig}.
The LQ signal is reconstructed using mainly energy measurements from the
electromagnetic (ECAL) and hadronic (HCAL) calorimeters, while tracker information 
is used primarily to identify and select electrons. 
The reconstruction and identification of high energy electrons and jets are described, respectively, 
in Sections \ref{sec:electrons} and \ref{sec:jet}.
Techniques used to identify the LQ signal include a search for features in the invariant mass
spectrum of the electron-jet pairs, and data-driven techniques to estimate 
and separate the signal from dominant background processes.
In Section \ref{sec:eventSelection}, the analysis strategy and the efficiencies of the chosen selection criteria 
are discussed. Data-driven techniques for estimating background are presented in Section \ref{sec:bkgStudy}.
Finally, the procedure for extraction of signal and the CMS discovery potential in this channel are described,
respectively, in Section \ref{sec:signalExtraction} and \ref{CMSpotential}.


%Experiments at the Tevatron have placed the current lower mass limit of the first generation scalar leptoquark at 251 GeV (assuming a branching ratio of 1 into electrons) \cite{d02007}.  This study includes 
%signal samples produced at a range of masses from 250 GeV to 1 TeV.  Monte Carlo simulation studies show the reach of CMS to be leptoquarks of masses around 1 TeV.  Assuming a mass of the leptoquark at the 
%current limit the studies show a $5 \sigma$ discovery possibility with just above $5pb^{-1}$ of integrated luminosity.  With the full statistics used in this study of  $100pb^{-1}$  a leptoquark of mass close 
%to 500 GeV could be discovered with a significance of $5 \sigma$.
%Limits if no discovery?

\section{Monte Carlo Samples} \label{sec:MCSamples}

%By far the dominant process is that of gluon-gluon 
%fusion, as can been seen in table~\ref{tab:CTEQ6} where the leading order and next to leading order cross sections are shown, 
%as well as the ratio of the processes contributing to the total cross section.  
%A more detailed description of the calculations of these numbers can be found in ~\cite{Kramer}.

%  \begin{table}[htb]
%    \caption{\small \sl Cross Section and K-factors using CTEQ6L1 and CTEQ6M parton densities for LQ productions at a variety of LQ masses\cite{Kramer} }
%    \label{tab:CTEQ6}
%    \begin{center}
%      \begin{tabular}{|l|cccc|} \hline
%           $M_{LQ}$ (GeV) & $\sigma_{LO}$ & $\sigma_{NLO}$ & gg:qq:gq & K \\ \hline
%        200  & $0.500\times10^{2}$ &  $0.742\times10^{2}$ &  $0.94:0.05:0.01$ &  $1.48$  \\ \hline
%        400  & $0.140\times10^{1}$ &  $0.224\times10^{1}$ &  $0.91:0.10:-0.01$ &  $1.60$  \\ \hline
%        600  & $0.135$ &  $0.225$ &  $0.88:0.15:-0.03$ &  $1.67$  \\ \hline
%        800  & $0.219\times10^{-1}$ &  $0.378\times10^{-1}$ &  $0.84:0.19:-0.03$ &  $1.73$  \\ \hline
%        1000  & $0.471\times10^{-2}$ &  $0.836\times10^{-2}$ &  $0.82:0.22:-0.04$ &  $1.77$  \\ \hline
%      \end{tabular}
%    \end{center}
%  \end{table}

MC signal samples are generated with leptoquark masses ranging from 250 GeV to 1 TeV and $\beta=1$. 
At the LHC, the three main processes that lead to leptoquark pair production are gluon-gluon fusion, 
quark-antiquark annihilation, and the higher order process of quark-gluon fusion. In the mass range to be investigated gluon-gluon fusion dominates. 

The "Full Simulation" (FullSim) signal samples are produced using 
CMSSW\_2\_1\_6 (PYTHIA version 6.227 and GEANT 4) 
These samples were included in the official "Summer 08" production by the CMS Generators group.
These signal events were generated for 100 $\mbox{pb}^{-1}$ samples with the accepted mis-calibration and alignment 
of the detectors. Table~1
%\ref{tab:NumEvents} 
shows the number of events and the cross sections at leading order (LO) for samples generated at different LQ mass.

%Additionally, samples produced with the Fast Simulation (FastSim) in CMSSW version 1\_6\_9 are 
%studied in order to validate the quality of the FastSim process. The FastSim uses a parameterization 
%of the detector response instead of using the full GEANT-based simulation to determine the detector response.
%The FastSim samples are created with the same mis-calibration and mis-alignment scheme as the FullSim samples. 
%%Table ~\ref{tab:FastVsFullSim} shows the relative time and size advantage of using the FastSim over the FullSim for leptoquark event production.  
%The use of the FastSim would be highly beneficial to this analysis, as the production of events with such large EM showers, and large numbers of particles generated 
%are quite time consuming in the FullSim.  
%%The comparison of the MC events generated with FullSim to the FastSim shows an agreement at few \% level in 
%%the shape of reconstructed quantities, and some discrepancies around 5-10\% in the reconstruction efficiencies. 
%Plots comparing various reconstructed quantities are included in some sub-sections of this note.  
%The main results of this analysis, including selection efficiencies and discovery potential are obtained with FullSim samples.

%\begin{table}[htb]
%  \label{tab:NumEvents}
%  \begin{center}
%    \begin{tabular}{|l|ccc|} \hline
%%      & \multicolumn{4}{c|}{Leptoquark Mass (GeV)} \\ 
%      LQ mass (GeV) & FullSim & FastSim & $\sigma_{LO}$ (pb) \\ \hline
%      250  & 1k  &  20k  &  14   \\
%      400  & 1k  &  20k  & 1.2    \\
%      650  & 2.5k &  20k  & 0.076    \\
%      1000 & 1k  &  20k  & 0.0047    \\
%      \hline
%    \end{tabular}
%    \caption{\small \sl Number of generated events and cross sections at LO, for LQ MC samples with various masses.  
%      FullSim samples made with CMSSW\_1\_4\_12 + CMSSW\_1\_6\_7, FastSim samples made with CMSSW\_1\_6\_9.}
%  \end{center}
%\end{table}
%
\begin{table}[htb]
  \label{tab:NumEvents}
  \begin{center}
    \begin{tabular}{|l|cc|} \hline
%      & \multicolumn{4}{c|}{Leptoquark Mass (GeV)} \\ 
      LQ mass (GeV) & FullSim &  $\sigma_{LO}$ (pb) \\ \hline
      250  & 50k  & 14 \\
      400  & 50k  & 1.2 \\
       \hline
    \end{tabular}
    \caption{\small \sl Number of generated events and cross sections at LO, for LQ MC samples with various masses.  
      FullSim samples made with CMSSW\_2\_1\_6 }
  \end{center}
\end{table}

 
%from : twiki.cern.ch/twiki/bin/view/CMS/CSA07Physics
%https://twiki.cern.ch/twiki/bin/view/CMS/GeneratorProduction2007CSA07


%  \begin{table}[htb]
%    \caption{\small \sl Average time needed to produce one event and size of one event for samples in FastSim and FullSim}
%    \label{tab:FastVsFullSim}
%    \begin{center}
%      \begin{tabular}{|l|c|cccc|} \hline
%               & \multicolumn{1}{c|}{FullSim} & \multicolumn{4}{c|}{FastSim} \\ 
%        Leptoquark Mass (GeV) &  650 & 250 & 400 & 650 & 1000  \\ \hline
%        Time per Event (sec) & 330  &  X  & 2.1  & 2.4  &  2.7     \\
%        Size of Event (kbytes)  & 3.5M (Reco) & Y & 116k &  123k & 132k  \\ \hline
%
%      \end{tabular}
%    \end{center}
%  \end{table}

%twiki.cern.ch/twiki/bin/view/CMS/FastSimPhysicsValidationInfo
%twiki.cern.ch/twiki/bin/view/CMS/SWGuideFastSimulation

The SM background is taken from the Summer08 official production samples,
which consists of more than 150M events of various SM processes roughly representing the first 100 pb$^{-1}$ of LHC data.
All events are generated and simulated with CMSSW\_2\_1\_6. 
The Summer08 samples used for this analysis are listed below:
\begin{itemize}
%
\item $t\bar{t}$+jet events (for N<5), generated using MADGRAPH \cite{Mangano:2002ea} with a 
LO-NLO K correction factor of 1.85 applied 
to the LQ cross section;\footnote{The K factor of 1.85 applied to ttbar events is a default in the software provided by CMS for 
 these data samples.  It is from a MCFM NLO vs LO study.}
%
\item $Z/\gamma*$+jet events (for N<6), generated using MADGRAPH
;  
%
\item $W$+Njet events (for N<6), generated using MADGRAPH, $ 0 < P_{T}^{W} < 300 $ GeV, $W$ decaying into leptons;  
%
\item QCD multi-jet events, generated with PYTHIA, inclusive production in $P_{T}^{jet}$ bins from 0 to $\infty$;  
%
\item $\gamma$+jet events, generated with PYTHIA, $ 0 < P_{T}^{jet} < 7000 $ GeV;  
%
\item $WW$, $WZ$, $ZZ$ events, generated with PYTHIA, inclusive production (all decays).
\end{itemize} 

%In the CSA07 production there were three flavors of Standard Model event combinations, two of which were used in this analysis: 
%the {\sl Chowder} contains ALPGEN W+jet, Z+jet and tt+jet samples, and the {\sl Gumbo} which contains QCD, Photon+jets, MinBias.  
%The event samples for each process within the soups were produced with numbers of events corresponding to a variety of integrated luminosities. 
%In order to scale the sample sizes correctly a central module that calculates the event weights has been used ({\sl CSA07EventWeightProducer.cc}) . 

%\end{document}
