%\documentclass{cmspaper}
%\begin{document}

\section{CMS Potential} \label{CMSpotential}

%Discovery luminosity

%5 sigma discovery

%Upper limits in absence of signal

%\subsection{Discovery Luminosity}

In order to estimate the integrated luminosity needed by CMS to claim 
a discovery in the electron channel of the first generation leptoquark
several MC experiments with signal and background events are produced
as described in section \ref{sec:signalExtraction}.

The significance of each experiment is extracted using the log-likelihood estimator 
\begin{displaymath}
S_L \equiv \sqrt{2\cdot ln{(L_{s+b}/L_{b})}}~\mathrm{,}
\end{displaymath}
where $L_{s+b}$ and $L_b$ are the maximum likelihood values for the fit to
the signal plus background and background only hypothesis, respectively.

A set of MC experiments is generated for a leptoquark mass and a value of 
the integrated luminosity.
The distribution of $S_L$ of the set of experiments is approximately gaussian 
and its median is used as the estimate of the significance.
The procedure is repeated for leptoquark masses of 250, 400 and 650~GeV and
different values of the integrated luminosity. The results are shown in
Figure~\ref{fig:sign_vs_Lint_sysR}, which includes also the effect of the systematic 
uncertainties discussed in the next section. 

% \begin{figure}
%   \begin{center}
%     \resizebox{7cm}{!}{\includegraphics{plots/significanceDistribution.eps}}
%     \caption{\small \sl significanceDistribution}
%     \label{fig:significanceDistribution}
%   \end{center}
% \end{figure}

% \begin{figure}
%   \begin{center}
%     \resizebox{10cm}{!}{\includegraphics{plots/significanceVsIntLum.eps}}
%     \caption{\small \sl Significance of the discovery as a function of integrated luminosity.
%     The curves represents fits to the points of a certain leptoquark mass using a 
%     square root function. The horizontal line represents a 5 sigma discovery.}
%     \label{fig:significanceVsIntLum}
%   \end{center}
% \end{figure}

\subsection{Systematic uncertainties}

%The impact of systematic effects on the discovery potential and the determination 
%of the experiment sensitivity are not yet included in this study.

%A study for determining the sensitivity of the experiment will be performed.
%It will be used to set upper limits in case a signal is not observed. 
%It will be used an approach similar to the one used in \ref{highmassToMuons}.
%The same tools to generate MC experiments and perform log-likelihood fits will be used.

%Systematics: Luminosity(+_10\%), JES (+_10\%), ratio ttbar/Zgammajets, etc.

The method used to discriminate between signal and background events and extract the significance 
of a possible observed leptoquark signal 
relies on the knowledge of the shape of the signal and background distributions 
by fitting the $M_{ej}$ distribution of the eejj sample.
The shape of the background distributions will be determined from the data as 
described in section~\ref{sec:bkgStudy}.
As a consequence, once real data is available, the uncertainty on the knowledge of quantities 
such as the integrated luminosity, the efficiency of the signal selection, the signal cross section, 
and the magnitude of the total background cross section, will not have an impact on the 
determination of the significance.
On the other hand, uncertainties that affect the knowledge of the shape of the distributions of 
the total background and the signal will have an impact on the significance and have to be 
accounted for as systematic uncertainties.

The value of $R$, defined in section \ref{sec:signalExtraction} as 
the ratio between the number of $t\bar{t}$ and $Z/\gamma$+jets events in the final sample,
directly affects the shape of the total background distribution used in the fit by changing the
mixture of the two backgrounds. 
Hence, the impact of the uncertainty on its value has to be included in the significance systematics.
The current knowledge, $R_{MC}$, of $R$ comes from the Monte Carlo
study and may introduce a deviation $\delta \equiv R/R_{MC}$ from the true value of $R$
due, for example, to imperfect knowledge of the relative magnitude of the two background 
cross sections and respective selection efficiencies.

Once real data is available there will be a way to infer the departure of $\delta$ from unity.
If $\tilde{R}$ is defined as the ratio between the number of $t\bar{t}$ and $Z/\gamma$+jets events in the
control samples described in section~\ref{sec:bkgStudy} and $\tilde{R}_{MC}$ is its
estimate from Monte Carlo, it is reasonable to expect that the main reasons that
make $\delta$ deviate from unity do affect $\tilde{R}/\tilde{R}_{MC}$ in the same way.
This is because the largest uncertainties of the MC are expected to come from the estimate 
of the cross section of the $t\bar{t}$ and $Z/\gamma$+jets processes, 
which affects equally $\tilde{R}_{MC}$ and $R_{MC}$.
Therefore, it can be concluded that $\delta \simeq \tilde{R}/\tilde{R}_{MC}$ and, once real data is available
to calculate $\tilde{R}$ from the control samples of $t\bar{t}$ and $Z/\gamma$+jets, this estimate 
of $\delta$ will be used to improve the knowledge of $R$ from the current estimate, $R_{MC}$, 
to $R_{data+MC} \equiv \delta \cdot R_{MC}$.
The significance fit will then be done using $R_{data+MC}$. 

The uncertainty on the estimate of $\delta$ given by $\tilde{R}/\tilde{R}_{MC}$ will be 
dominated by the statistical error of $\tilde{R}$ from the number of events in the control samples. 
This will propagate to give an uncertainty $\Delta R_{data+MC}$ on $R_{data+MC}$ and will
be larger for small integrated luminosities. 
Currently, $\Delta R_{data+MC}$ is estimated using the CSA07 control samples.
The fit to extract the significance is here repeated changing the value of $R$ by 
$\pm\Delta R_{data+MC}$ from its central value $R_{MC}$ \footnote{Not having real data yet, 
exact correctness of the MC is currently assumed by setting $\delta=1$, 
making effectively $R_{data+MC}=R_{MC}$.}, and the results are shown in 
Figure~\ref{fig:sign_vs_Lint_sysR}.
The systematic uncertainty due to $R$ is larger, for the same integrated luminosity, for a
LQ mass of 250~GeV. This is due to the topology of the $M_{ej}$ distribution 
of the signal that, at small LQ masses, peaks where the background does.
For higher values of the LQ mass, the significance determination seems to be little 
sensible to variations of $R$. This may be understood by considering the increased separation
between the main features of the signal and backgound distributions, and the relative similarity 
between the two backgound shapes: 
the mixture between $t\bar{t}$ and $Z/\gamma$+jets events may vary, but
the mixture between total background and signal does little.
As an example, for the point at $M_{LQ}=400$~GeV, integrated luminosity=100~$pb^{-1}$ and significance=12.7
the uncertainty  $\Delta R_{data+MC}$ is about 15\%, but this determines a change of 
significance of only 1\%.

The fact that, apart from the small LQ mass region and low luminosity scenario where special care
will have to be applied, the uncertainty 
of $R$ tends to be ``absorbed'' in the fitting procedure and have a limited impact on the 
significance determination indicates that this systematics is reasonably under control. 
In addition, it reassures that once the real data is available and the value of the correction 
factor $\delta$, currently set to unity, can be estimated as described above and used in the fit,
even a significant departure from unity will not require a drastic re-assessment of
the current estimate of the luminosity needed for discovery.
The uncertainty on the knowledge of the signal shape used in the fit may also produce a systematic
uncertainty on the significance. A study of this effect will have to be performed.

 \begin{figure}
   \begin{center}
     \resizebox{10cm}{!}{\includegraphics{plots/sign_vs_Lint_sysR.eps}}
     \caption{\small \sl Significance of the discovery as a function of the integrated luminosity.
       For each LQ mass, the fit to extract the significance is repeated varying the value of $R$
       (the ratio between the number of $t\bar{t}$ and $Z/\gamma$+jets events in the final sample)
       of an amount $\pm\Delta R_{data+MC}$ from its central value $R_{data+MC}$ 
       (see text for details).
       Apart from a couple of points at a LQ mass of 650~GeV, the higher curves correspond to
       the higher value of $R$. 
       The horizontal line represents a 5 sigma discovery.}
     \label{fig:sign_vs_Lint_sysR}
   \end{center}
 \end{figure}

In the case of setting an upper limit in the case of absence of signal, a larger number of systematic
uncertainties will have to be considered as they do have a direct impact on it. 
Such uncertainties include the one on the integrated luminosity, the detector efficiency and 
resolution, the jet energy scale, and theoretical uncertainties such as those on parton distributions
and higher-order corrections on the cross sections. Different scenarios of levels of background
and signal will have to be studied. The procedure for setting an upper limit and the study of the 
systematics involved is not yet included in this version of the analysis.



%\end{document}
