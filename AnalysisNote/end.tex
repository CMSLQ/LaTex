%\documentclass{cmspaper}
%\begin{document}

\section{Conclusion}

A MC-based analysis of the pair production of first generation leptoquarks that decay to
an electron and a jet has been presented.
The analysis strategy and detector performance description corresponds to a 100$~pb^{-1}$ scenario.
Standard CMS techniques are used for electron and jet identification. 
%Electron efficiencies will be determined from data using the Tag$\&$Pro [REF] technique. 
A cut-based event selection has been applied and its optimization study will come in a future upgrade
of the analysis.
Several background sources have been studied and only two of them provide a significant contribution
after event selection. 
Data-driven techniques to understand the characteristics of these contributions have been developed 
and studied on the CSA07 samples.

The signal component is extracted with a log-likelihood fit that does not rely on precise knowledge
of the level of the background but rather by exploiting the different shapes of the signal and background
distributions. 
The discovery potential of this analysis has been determined using a likelihood ratio estimator. 
This has shown that, for leptoquark masses just above the Tevatron exclusion limit of 290~GeV
\cite{d02008}, an early discovery is possible with a few $pb^{-1}$ of data.
With an integrated luminosity of 100~$pb^{-1}$, discovery should be possible up
to LQ mass of about 500 GeV.
The procedure for setting an upper limit in case of absence of signal will be studied 
in the next upgrade of the analysis.



%To say:
%- room for improvement, however striking signal features ...
%- optimization of selection to be done
%- data-driven method for bkg used
%- TagAndPro for electron effic determination
%- systematics for 5sigma discovery, more important for sensitivity curve, will use approach of CMS-AN-2007-038



%\end{document}
