%\documentclass{cmspaper}
%\begin{document}

\section{Conclusion}

The search for pair production of first generation leptoquarks that decay to
an electron and a jet has been studies using a MC simulation.
The analysis strategy and detector performance assume an integrated luminosity of 100 pb$^{-1}$ and $pp$ collisions 
at $\sqrt{s}=10$ TeV.
Standard CMS techniques are used for electron and jet identification. 
%Electron efficiencies will be determined from data using the Tag$\&$Pro [REF] technique. 
An optimized cut-based event selection has been applied.
The main expected background sources have been studied and two of them provide 
a significant contribution after event selection. 
Data-driven techniques to understand the characteristics of these contributions have been developed.

The discovery and exclusion potential in the channel with two electrons plus two jets has 
been determined using two statistical estimators suited for a counting experiment in Poissonian regime.
Effect of the main systematic uncertainties has been studied and taken into account in the final 
results. This study suggests that, 
for leptoquark masses just above the Tevatron exclusion limit of 256~GeV assuming $\beta=1$, 
an early discovery could be possible with a few $pb^{-1}$ of data.
With an integrated luminosity of 100~$pb^{-1}$, discovery should be possible up
to leptoquark mass of about 500, 440, and 280 GeV assuming, respectively, 
$\beta=1$, $0.5$, and $0.1$. 
In absence of evidence of a signal, the existence of a scalar leptoquark 
with mass lower than about 570, 505, and 350 GeV, assuming respectively 
$\beta=1$, $0.5$, and $0.1$, can be excluded with 100~pb$^{-1}$ of data.

%To say:
%- room for improvement, however striking signal features ...
%- optimization of selection to be done
%- data-driven method for bkg used
%- TagAndPro for electron effic determination
%- systematics for 5sigma discovery, more important for sensitivity curve, will use approach of CMS-AN-2007-038



%\end{document}
