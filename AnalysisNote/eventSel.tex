%\documentclass{cmspaper}
%\begin{document}

\section{Event Selection} \label{sec:eventSelection}

% Analysis strategy (M dist, bump search)
% Optimize signal, minimize bkgd
% Plots of selection variables
% (optimization of selection criteria)
% List of selection criteria (Mee, St, etc.)

The basic strategy to identify the existence of a particle that decays to a jet and an electron 
is to study the invariant mass of the electron-jet pairs in the events. 
The signal of a leptoquark would appear as a bump in the distribution of this invariant mass.
The decay of a particle to a jet and a lepton is forbidden in the standard model.  
Thus, a peak at the mass of the leptoquark would be the only resonance in this distribution. 
The electron-jet combinations of the background events have a different shape than that 
of the leptoquark resonance. 
Nonetheless, the number of background events could dominate over the number of signal events.  
The following selections have been made in order to minimize the number of background events 
selected while retaining a high signal efficiency.

The decay of a leptoquark and an anti-leptoquark produces two high energy jets and two high energy 
electrons.  
The online selection of candidate events is made by the HLT triggers EMHighEt and EMVeryHighEt 
described in section~\ref{sec:trig} while the offline identification of electrons and jets 
proceeds as described in sections~\ref{sec:electrons} and \ref{sec:jet}, respectively.
The offline selection of the eejj sample continues with the following cuts:
%
\begin{enumerate}
\item at least 2 isolated electrons, one required to have $P_T>85$~GeV, 
and the other $P_T>30$~GeV 
\item at least 2 jets, both required to have $P_T>50$~GeV
\item $M_{ee}>100$~GeV
\item $S_T\equiv P_T(e_1)+P_T(e_2)+P_T(j_1)+P_T(j_2)>400$~GeV
\end{enumerate}
%
The $P_T$ threshold for one electron in cut 1 is set to 5~GeV above the threshold of the $EMHighEt$ trigger 
in order to increase the confidence that the trigger was actually fired by a real electron in all events.
The other electron will have a $P_T>30$~GeV, as it is implicit in the definition of electron. 
Cut 3, where $M_{ee}$ is the invariant mass of the electron pair, removes background events from 
$Z/\gamma$+jets events as shown in figure~\ref{fig:Mee_St_distributions}-left.
Cut 4, where the variable $S_T$ is defined as the scalar sum of the transverse momenta of the 
2 electrons and 2 jets, is applied following the approach of the experiment $D0$ in 
\cite{Abazov:2001mx}. In that paper, an event selection optimization as a function of
the combinations of several kinematic variables has shown that $S_T$ is the most powerful one 
and little is gained by adding cuts on other variables. The distribution of $S_T$ for the present
analysis is shown in figure~\ref{fig:Mee_St_distributions}-right.

Once the eejj sample is built, there are two ways to combine two electrons and two jets to make two electron-jet pairs. 
For each event, the combination with the minimum difference, $\Delta M_{ej}$, between the invariant masses, $M_{ej}$, 
of the two electron-jet pairs is chosen. 
The resulting $M_{ej}$ distribution is shown in figure~\ref{fig:Mej_allComb} for the signal and remaining backgrounds. 
A detailed study of the optimization of the selection criteria, and of alternative algorithms to identify the best candidate 
electron-jet combinations, will be performed in future upgrades of this analysis.

\begin{figure}[htbp]
  \begin{center}
    \begin{tabular}{cc}
      \resizebox{7.5cm}{!}{\includegraphics{plots/Mee_distribution.eps}} &
      \resizebox{7.5cm}{!}{\includegraphics{plots/St_distribution.eps}} \\
    \end{tabular}
    \caption{\small \sl Left: invariant mass of the electron pair, $M_{ee}$. 
             Right: scalar sum of the $P_T$ of the 2 leading electrons and 2 leading jets. 
	     In each histogram, the distributions for the signal (for leptoquark mass 250 and 650~GeV) and the 
	     relevant backgrounds are shown after applying all cuts except the one involving the 
	     plotted variable. 
	     The background histograms are summed on top of each other.
	     The histogram labeled with ``Others'' contains: 1) for $M_{ee}$, $W$+jets (33\%), 
	     $WW$ (11\%), $ZZ$ (20\%), and $WZ$ (36\%) events; 2) for $S_{T}$, $W$+jets (57\%), 
	     $WW$ (19\%), $ZZ$ (8\%), $WZ$ (14\%), and $\gamma$+jet (2\%) events.
	     With the available MC statistics, no QCD multi-jet events pass 
	     the selection.}
	     %Arrows indicate the cut subsequently applied.
    \label{fig:Mee_St_distributions}
  \end{center}
\end{figure}


\begin{figure}[htbp]
  \begin{center}
    \resizebox{10cm}{!}{\includegraphics{plots/Mej_best_Meecut_Stcut.eps}}
    \caption{\small \sl Distribution of the invariant mass, $M_{ej}$, 
      of the electron-jet pairs 
      with smaller $\Delta M_{ej}$
      for signal (at 250~GeV and 650~GeV LQ mass) and backgrounds. 
      The complete event selection has been applied.
      The background histograms are summed on top of each other.
      The histogram labeled with ``Others'' contains $W$+jets (65\%), 
      $WW$ (20\%), $ZZ$ (8\%), and $WZ$ (7\%) events. 
      With the available MC statistics, no QCD multi-jet and $\gamma$+jet events pass 
      the selection.}
    \label{fig:Mej_allComb}
  \end{center}
\end{figure}


Tables from \ref{tab:selection_effic_250} to \ref{tab:selection_effic_1000} show the efficiency 
of each selection cut for signal events, using FullSim samples at different LQ masses (from 250 GeV to 1 TeV).
Tables \ref{tab:selection_effic_ttbar} and \ref{tab:selection_effic_zjet} show the efficiency of each selection cut 
for $t\bar{t}$ and $Z/\gamma$+jet events, which are the dominant backgrounds in the eejj sample. 
A summary of the number of selected signal and background events expected in 100 pb$^{-1}$ of data 
is reported in Table \ref{tab:EventSelSummary}. 
The overall signal selection efficiencies are around 30-50\% for the LQ masses investigated. 
After the described event selection, the dominant background contributions 
are $t\bar{t}$ and $Z/\gamma$+jets. Data-driven techniques to estimate these backgrounds
are discussed in the next section.


\begin{table}[htbp]
\begin{center}
\begin{tabular}{|c|c|c|c|}
\hline
\hline
 & $N_{ev}$ $100pb^{-1}$ & $N_{ev}$ & $\varepsilon$ \\
\hline
\hline

no cut &1400 $\pm$ 44& 1000 & - \\
+ hlt (HE or VHE) &1121 $\pm$ 40 & 801 & 8.01e-01 $\pm$ 1.2e-02\\
+ 2 ele (ID) $P_{T} >30$ GeV &644 $\pm$ 30 & 460 & 5.74e-01 $\pm$ 3.3e-02\\
+ 2 ele (ID+Iso) $P_{T} >30$ GeV &592 $\pm$ 29 & 423 & 9.20e-01 $\pm$ 6.1e-02\\
+ 2e, 2j $P_{T}^{ele}$ ($P_{T}^{jet}$) $>$30(50) GeV &508 $\pm$ 27& 363 & 8.58e-01 $\pm$ 6.1e-02\\
+ $P_{T}$ $1^{st}$ ele $>$ 85 GeV &506 $\pm$ 27& 362 & 9.97e-01 $\pm$ 7.4e-02\\
+ $M_{ee} >100$ GeV& 455 $\pm$ 25& 325 & 8.98e-01 $\pm$ 6.8e-02\\
+ $S_{T} >400$ GeV &425 $\pm$ 24& 304 & 9.35e-01 $\pm$ 7.4e-02\\
\hline

full selection efficiency& &  & 3.04e-01 $\pm$ 1.4e-02\\
\hline
\end{tabular}
\end{center}
\caption{\small \sl LQ sample with $M_{LQ}=250$ GeV: the first column lists the selection sequence, $N_{ev}$ $100pb^{-1}$ is the number of selected events in $100pb^{-1}$, $N_{ev}$ is the absolute number of selected events (unweighted for the cross section), $\varepsilon$ is the relative selection efficiency with respect to the previous cut. The full selection efficiency is shown in the last line.}
\label{tab:selection_effic_250}
\end{table}

%old
%\begin{table}[htbp]
%\begin{center}
%\begin{tabular}{|c|c|c|c|}
%\hline
%\hline
% & $N_{ev}$ $100pb^{-1}$ & $N_{ev}$ & $\varepsilon$ \\
%\hline
%\hline
%no cut &1400.0 $\pm$ 44.3& 1000.0 & - \\
%+ hlt (HE or VHE) &1124.2 $\pm$ 39.7& 803.0 & 8.0e-01 $\pm$ 1.3e-02\\
%%+  &1121.4 $\pm$ 39.6& 801.0 & 1.0e+00 $\pm$ 5.0e-02\\
%+ 2 ele (no Iso) pT $>30.0$ GeV &644.0 $\pm$ 30.0& 460.0 & 5.7e-01 $\pm$ 3.4e-02\\
%+ 2 ele (Iso) pT $>30.0$ GeV &596.4 $\pm$ 28.9& 426.0 & 9.3e-01 $\pm$ 6.2e-02\\
%+ 2e, 2j pT$_{ele}$(pT$_{jet}$) $>$30.0(50.0) GeV &512.4 $\pm$ 26.8& 366.0 & 8.6e-01 $\pm$ 6.1e-02\\
%+ pT 1st ele $>$ 85.0 GeV &511.0 $\pm$ 26.7& 365.0 & 1.0e+00 $\pm$ 7.4e-02\\
%+ Mee $<80.0$ GeV or Mee $>100.0$ GeV&499.8 $\pm$ 26.5& 357.0 & 9.8e-01 $\pm$ 7.3e-02\\
%+ St $>400.0$ GeV &462.0 $\pm$ 25.4& 330.0 & 9.2e-01 $\pm$ 7.1e-02\\
%\hline
%full selection efficiency& &  & 4.1e-01 $\pm$ 2.7e-02\\
%\hline
%\end{tabular}
%\end{center}
%\caption{\small \sl LQ sample with $M_{LQ}=250$ GeV: the first column lists the selection sequence, $N_{ev}$ $100pb^{-1}$ is the number of selected events in $100pb^{-1}$, $N_{ev}$ is the absolute number 
%of selected events (unweighted for the cross section), $\varepsilon$ is the relative selection efficiency with respect to the previous cut. 
%The overall selection efficiency is shown in the last line.}
%\label{tab:selection_effic_250}
%\end{table}



\begin{table}[htbp]
\begin{center}
\begin{tabular}{|c|c|c|c|}
\hline
\hline
 & $N_{ev}$ $100pb^{-1}$ & $N_{ev}$ & $\varepsilon$ \\
\hline
\hline

no cut &120.0 $\pm$ 3.8 & 1000 & - \\
+ hlt (HE or VHE) &111.1 $\pm$ 3.6& 926 & 9.26e-01 $\pm$ 0.8e-02\\
+ 2 ele (ID) $P_{T} >30$ GeV &67.1 $\pm$ 2.8& 559 & 6.04e-01 $\pm$ 3.2e-02\\
+ 2 ele (ID+Iso) $P_{T} >30$ GeV &59.5 $\pm$ 2.7& 496 & 8.87e-01 $\pm$ 5.5e-02\\
+ 2e, 2j $P_{T}^{ele}$ ($P_{T}^{jet}$) $>$30(50) GeV &55.4 $\pm$ 2.6& 462 & 9.31e-01 $\pm$ 6.0e-02\\
+ $P_{T}$ $1^{st}$ ele $>$ 85 GeV &55.4 $\pm$ 2.6& 462 & 1.00e+00 $\pm$ 6.6e-02\\
+ $M_{ee} >100$ GeV&52.7 $\pm$ 2.5& 439 & 9.50e-01 $\pm$ 6.3e-02\\
+ $S_{T} >400$ GeV &52.4 $\pm$ 2.5& 437 & 9.95e-01 $\pm$ 6.7e-02\\
\hline

full selection efficiency& &  & 4.37e-01 $\pm$ 1.57e-02\\
\hline
\end{tabular}
\end{center}
\caption{\small \sl LQ sample with $M_{LQ}=400$ GeV: the first column lists the selection sequence, $N_{ev}$ $100pb^{-1}$ is the number of selected events in $100pb^{-1}$, $N_{ev}$ is the absolute number of selected events (unweighted for the cross section), $\varepsilon$ is the relative selection efficiency with respect to the previous cut. The full selection efficiency is shown in the last line.}
\label{tab:selection_effic_400}
\end{table}

%old
%\begin{table}[htbp]
%\begin{center}
%\begin{tabular}{|c|c|c|c|}
%\hline
%\hline
% & $N_{ev}$ $100pb^{-1}$ & $N_{ev}$ & $\varepsilon$ \\
%\hline
%\hline

%no cut &120.0 $\pm$ 3.8& 1000.0 & - \\
%+ hlt (HE or VHE) &111.1 $\pm$ 3.7& 926.0 & 9.3e-01 $\pm$ 8.3e-03\\
%%+  &111.1 $\pm$ 3.7& 926.0 & 1.0e+00 $\pm$ 4.6e-02\\
%+ 2 ele (no Iso) pT $>30.0$ GeV &67.1 $\pm$ 2.8& 559.0 & 6.0e-01 $\pm$ 3.2e-02\\
%+ 2 ele (Iso) pT $>30.0$ GeV &61.1 $\pm$ 2.7& 509.0 & 9.1e-01 $\pm$ 5.6e-02\\
%+ 2e, 2j pT$_{ele}$(pT$_{jet}$) $>$30.0(50.0) GeV &56.8 $\pm$ 2.6& 473.0 & 9.3e-01 $\pm$ 5.9e-02\\
%+ pT 1st ele $>$ 85.0 GeV &56.8 $\pm$ 2.6& 473.0 & 1.0e+00 $\pm$ 6.5e-02\\
%+ Mee $<80.0$ GeV or Mee $>100.0$ GeV&55.4 $\pm$ 2.6& 462.0 & 9.8e-01 $\pm$ 6.4e-02\\
%+ St $>400.0$ GeV &55.2 $\pm$ 2.6& 460.0 & 1.0e+00 $\pm$ 6.6e-02\\
%\hline

%full selection efficiency& &  & 5.0e-01 $\pm$ 2.8e-02\\
%\hline
%\end{tabular}
%\end{center}
%\caption{\small \sl LQ sample with $M_{LQ}=400$ GeV: the first column lists the selection sequence, $N_{ev}$ $100pb^{-1}$ is the number of selected events in $100pb^{-1}$, $N_{ev}$ is the absolute number 
%of selected events (unweighted for the cross section), $\varepsilon$ is the relative selection efficiency with respect to the previous cut. 
%The overall selection efficiency is shown in the last line.}
%\label{tab:selection_effic_400}
%\end{table}


\begin{table}[htbp]
\begin{center}
\begin{tabular}{|c|c|c|c|}
\hline
\hline
 & $N_{ev}$ $100pb^{-1}$ & $N_{ev}$ & $\varepsilon$ \\
\hline
\hline

no cut &7.20 $\pm$ 0.14& 2500 & - \\
+ hlt (HE or VHE) &6.94 $\pm$ 0.14& 2408 & 9.63e-01 $\pm$ 3.7e-02\\
+ 2 ele (ID) $P_{T} >30$ GeV &4.15 $\pm$ 0.11& 1440 & 5.98e-01 $\pm$ 1.9e-02\\
+ 2 ele (ID+Iso) $P_{T} >30$ GeV &3.58 $\pm$ 0.10 & 1243 & 8.63e-01 $\pm$ 3.3e-02\\
+ 2e, 2j $P_{T}^{ele}$ ($P_{T}^{jet}$) $>$30(50) GeV &3.45 $\pm$ 0.10& 1198 & 9.64e-01 $\pm$ 3.9e-02\\
+ $P_{T}$ $1^{st}$ ele $>$ 85 GeV &3.45 $\pm$ 0.10& 1198 & 1.00 $\pm$ 0.4e-01\\
+ $M_{ee} >100$ GeV&3.41 $\pm$ 0.10& 1184 & 9.88e-01 $\pm$ 4.0e-02\\
+ $S_{T} >400$ GeV &3.41 $\pm$ 0.10& 1184 & 1.00 $\pm$ 0.4e-01\\
\hline

full selection efficiency& &  & 4.74e-01 $\pm$ 9.99e-03\\
\hline
\end{tabular}
\end{center}
\caption{\small \sl LQ sample with $M_{LQ}=650$ GeV: the first column lists the selection sequence, $N_{ev}$ $100pb^{-1}$ is the number of selected events in $100pb^{-1}$, $N_{ev}$ is the absolute number of selected events (unweighted for the cross section), $\varepsilon$ is the relative selection efficiency with respect to the previous cut. The full selection efficiency is shown in the last line.}
\label{tab:selection_effic_650}
\end{table}


%\begin{table}[htbp]
%\begin{center}
%\begin{tabular}{|c|c|c|c|}
%\hline
%\hline
% & $N_{ev}$ $100pb^{-1}$ & $N_{ev}$ & $\varepsilon$ \\
%\hline
%\hline

%no cut &7.2 $\pm$ 0.1& 2500.0 & - \\
%+ hlt (HE or VHE) &6.9 $\pm$ 0.1& 2408.0 & 9.6e-01 $\pm$ 3.8e-03\\
%%+  &6.9 $\pm$ 0.1& 2408.0 & 1.0e+00 $\pm$ 2.9e-02\\
%+ 2 ele (no Iso) pT $>30.0$ GeV &4.2 $\pm$ 0.1& 1441.0 & 6.0e-01 $\pm$ 2.0e-02\\
%+ 2 ele (Iso) pT $>30.0$ GeV &3.7 $\pm$ 0.1& 1292.0 & 9.0e-01 $\pm$ 3.4e-02\\
%+ 2e, 2j pT$_{ele}$(pT$_{jet}$) $>$30.0(50.0) GeV &3.6 $\pm$ 0.1& 1247.0 & 9.7e-01 $\pm$ 3.8e-02\\
%+ pT 1st ele $>$ 85.0 GeV &3.6 $\pm$ 0.1& 1247.0 & 1.0e+00 $\pm$ 4.0e-02\\
%+ Mee $<80.0$ GeV or Mee $>100.0$ GeV&3.6 $\pm$ 0.1& 1241.0 & 1.0e+00 $\pm$ 4.0e-02\\
%+ St $>400.0$ GeV &3.6 $\pm$ 0.1& 1241.0 & 1.0e+00 $\pm$ 4.0e-02\\
%\hline

%full selection efficiency& &  & 5.2e-01 $\pm$ 1.8e-02\\
%\hline
%\end{tabular}
%\end{center}
%\caption{\small \sl LQ sample with $M_{LQ}=650$ GeV: the first column lists the selection sequence, $N_{ev}$ $100pb^{-1}$ is the number of selected events in $100pb^{-1}$, $N_{ev}$ is the absolute number 
%of selected events (unweighted for the cross section), $\varepsilon$ is the relative selection efficiency with respect to the previous cut. 
%The overall selection efficiency is shown in the last line.}
%\label{tab:selection_effic_650}
%\end{table}


\begin{table}[htbp]
\begin{center}
\begin{tabular}{|c|c|c|c|}
\hline
\hline
 & $N_{ev}$ $100pb^{-1}$ & $N_{ev}$ & $\varepsilon$ \\
\hline
\hline

no cut &0.47 $\pm$ 0.01& 1000 & - \\
+ hlt (HE or VHE) &0.45 $\pm$ 0.01& 967 & 9.67e-01 $\pm$ 0.5e-02\\
+ 2 ele (ID) $P_{T} >30$ GeV &0.27 $\pm$ 0.01& 583 & 6.03e-01 $\pm$ 3.1e-02\\
+ 2 ele (ID+Iso) $P_{T} >30$ GeV &0.21 $\pm$ 0.01& 455 & 7.80e-01 $\pm$ 4.8e-02\\
+ 2e, 2j $P_{T}^{ele}$ ($P_{T}^{jet}$) $>$30(50) GeV &0.21 $\pm$ 0.01& 443 & 9.74e-01 $\pm$ 6.5e-02\\
+ $P_{T}$ $1^{st}$ ele $>$ 85 GeV &0.21 $\pm$ 0.01& 443 & 1.00 $\pm$ 0.6e-01\\
+ $M_{ee} >100$ GeV&0.21 $\pm$ 0.01& 443 & 1.00 $\pm$ 0.6e-01\\
+ $S_{T} >400$ GeV &0.21 $\pm$ 0.01& 443 & 1.00 $\pm$ 0.6e-01\\
\hline

full selection efficiency& &  & 4.43e-01 $\pm$ 1.5e-02\\
\hline
\end{tabular}
\end{center}
\caption{\small \sl LQ sample with $M_{LQ}=1000$ GeV: the first column lists the selection sequence, $N_{ev}$ $100pb^{-1}$ is the number of selected events in $100pb^{-1}$, $N_{ev}$ is the absolute number of selected events (unweighted for the cross section), $\varepsilon$ is the relative selection efficiency with respect to the previous cut. The full selection efficiency is shown in the last line.}
\label{tab:selection_effic_1000}
\end{table}


%\begin{table}[htbp]
%\begin{center}
%\begin{tabular}{|c|c|c|c|}
%\hline
%\hline
% & $N_{ev}$ $100pb^{-1}$ & $N_{ev}$ & $\varepsilon$ \\
%\hline
%\hline

%no cut &0.50 $\pm$ 0.01& 1000.0 & - \\
%+ hlt (HE or VHE) &0.50 $\pm$ 0.01& 967.0 & 9.7e-01 $\pm$ 5.6e-03\\
%%+ hlt (HE or VHE) &0.5 $\pm$ 0.0& 967.0 & 1.0e+00 $\pm$ 4.5e-02\\
%+ 2 ele (no Iso) pT $>30.0$ GeV &0.30 $\pm$ 0.01& 583.0 & 6.0e-01 $\pm$ 3.2e-02\\
%+ 2 ele (Iso) pT $>30.0$ GeV &0.20 $\pm$ 0.01& 491.0 & 8.4e-01 $\pm$ 5.2e-02\\
%+ 2e, 2j pT$_{ele}$(pT$_{jet}$) $>$30.0(50.0) GeV &0.20 $\pm$ 0.01& 478.0 & 9.7e-01 $\pm$ 6.3e-02\\
%+ pT 1st ele $>$ 85.0 GeV &0.20 $\pm$ 0.01& 478.0 & 1.0e+00 $\pm$ 6.5e-02\\
%+ Mee $<80.0$ GeV or Mee $>100.0$ GeV&0.20 $\pm$ 0.01& 478.0 & 1.0e+00 $\pm$ 6.5e-02\\
%+ St $>400.0$ GeV &0.20 $\pm$ 0.01& 478.0 & 1.0e+00 $\pm$ 6.5e-02\\
%\hline

%full selection efficiency& &  & 4.9e-01 $\pm$ 2.8e-02\\
%\hline
%\end{tabular}
%\end{center}
%\caption{\small \sl LQ sample with $M_{LQ}=1000$ GeV: the first column lists the selection sequence, $N_{ev}$ $100pb^{-1}$ is the number of selected events in $100pb^{-1}$, $N_{ev}$ is the absolute number 
%of selected events (unweighted for the cross section), $\varepsilon$ is the relative selection efficiency with respect to the previous cut. 
%The overall selection efficiency is shown in the last line.}
%\label{tab:selection_effic_1000}
%\end{table}


\begin{table}[htbp]
\begin{center}
\begin{tabular}{|c|c|c|c|}
\hline
\hline
 & $N_{ev}$ $100pb^{-1}$ & $N_{ev}$ & $\varepsilon$ \\
\hline
\hline

+ hlt (HE or VHE) &2115.9 $\pm$ 9.7 & 48359 & - \\
+ 2 ele (ID) $P_{T} >30$ GeV &137.4 $\pm$ 2.5& 3139 & 6.49e-02 $\pm$ 1.2e-03\\
+ 2 ele (ID+Iso) $P_{T} >30$ GeV &105.4 $\pm$ 2.1 & 2408 & 7.67e-01 $\pm$ 2.0e-02\\
+ 2e, 2j $P_{T}^{ele}$ ($P_{T}^{jet}$) $>$30(50) GeV &49.6 $\pm$ 1.5& 1117 & 4.71e-01 $\pm$ 1.7e-02\\
+ $P_{T}$ $1^{st}$ ele $>$ 85 GeV &47.2 $\pm$ 1.4& 1063 & 9.52e-01 $\pm$ 4.1e-02\\
+ $M_{ee} >100$ GeV&40.2 $\pm$ 1.3& 905 & 8.52e-01 $\pm$ 3.9e-02\\
+ $S_{T} >400$ GeV &21.26 $\pm$ 0.98 & 476 & 5.29e-01 $\pm$ 3.0e-02\\
\hline
\end{tabular}
\end{center}
\caption{\small \sl $t\bar{t}$ sample: the first column lists the selection sequence, $N_{ev}$ $100pb^{-1}$ is the number of selected events in $100pb^{-1}$, $N_{ev}$ is the absolute number of selected events (unweighted for the cross section), $\varepsilon$ is the relative selection efficiency with respect to the previous cut.}
\label{tab:selection_effic_ttbar}
\end{table}


%\begin{table}[htbp]
%\begin{center}
%\begin{tabular}{|c|c|c|c|}
%\hline
%\hline
% & $N_{ev}$ $100pb^{-1}$ & $N_{ev}$ & $\varepsilon$ \\
%\hline
%\hline


%%no cut &0.0 $\pm$ 0.0& 0.0 & - \\
%%+ LQ skim &2391.5 $\pm$ 10.3& 54697.0 & inf $\pm$ 0.0e+00\\
%hlt (HE or VHE) &2115.9 $\pm$ 9.6& 48359.0 & 8.8e-01 $\pm$ 5.5e-03\\
%+ 2 ele (no Iso) pT $>30.0$ GeV &138.5 $\pm$ 2.5& 3164.0 & 6.5e-02 $\pm$ 1.2e-03\\
%+ 2 ele (Iso) pT $>30.0$ GeV &106.4 $\pm$ 2.2& 2432.0 & 7.7e-01 $\pm$ 2.1e-02\\
%+ 2e, 2j pT$_{ele}$(pT$_{jet}$) $>$30.0(50.0) GeV &50.2 $\pm$ 1.5& 1131.0 & 4.7e-01 $\pm$ 1.7e-02\\
%+ pT 1st ele $>$ 85.0 GeV &47.8 $\pm$ 1.5& 1076.0 & 9.5e-01 $\pm$ 4.1e-02\\
%+ Mee $<80.0$ GeV or Mee $>100.0$ GeV&45.0 $\pm$ 1.4& 1013.0 & 9.4e-01 $\pm$ 4.1e-02\\
%+ St $>400.0$ GeV &23.3 $\pm$ 1.0& 523.0 & 5.2e-01 $\pm$ 2.8e-02\\
%\hline
%\end{tabular}
%\end{center}
%\caption{\small \sl $t\bar{t}$ sample: the first column lists the selection sequence, $N_{ev}$ $100pb^{-1}$ is the number of selected events in $100pb^{-1}$, $N_{ev}$ is the absolute number 
%of selected events (unweighted for the cross section), $\varepsilon$ is the relative selection efficiency with respect to the previous cut.}
%\label{tab:selection_effic_ttbar}
%\end{table}



\begin{table}[htbp]
\begin{center}
\begin{tabular}{|c|c|c|c|}
\hline
\hline
 & $N_{ev}$ $100pb^{-1}$ & $N_{ev}$ & $\varepsilon$ \\
\hline
\hline

+ hlt (HE or VHE) &2466 $\pm$ 14& 30413 & - \\
+ 2 ele (ID) $P_{T} >30$ GeV &1356 $\pm$ 12& 13136 & 5.50e-01 $\pm$ 0.6e-02\\
+ 2 ele (ID+Iso) $P_{T} >30$ GeV &1329 $\pm$ 12& 12830 & 9.80e-01 $\pm$ 1.2e-02\\
+ 2e, 2j $P_{T}^{ele}$ ($P_{T}^{jet}$) $>$30(50) GeV &124.7 $\pm$ 3.0 & 1807 & 9.38e-02 $\pm$ 2.4e-03\\
+ $P_{T}$ $1^{st}$ ele $>$ 85 GeV &115.7 $\pm$ 2.9& 1694 & 9.28e-01 $\pm$ 3.2e-02\\
+ $M_{ee} >100$ GeV&10.82 $\pm$ 0.89& 157 & 9.35e-02 $\pm$ 8.0e-03\\
+ $S_{T} >400$ GeV &5.10 $\pm$ 0.60& 76 & 4.71e-01 $\pm$ 6.7e-02\\
\hline
\end{tabular}
\end{center}
\caption{\small \sl $Z/\gamma$ sample: the first column lists the selection sequence, $N_{ev}$ $100pb^{-1}$ is the number of selected events in $100pb^{-1}$, $N_{ev}$ is the absolute number of selected events (unweighted for the cross section), $\varepsilon$ is the relative selection efficiency with respect to the previous cut.}
\label{tab:selection_effic_zjet}
\end{table}

%\begin{table}[htbp]
%\begin{center}
%\begin{tabular}{|c|c|c|c|}
%\hline
%\hline
% & $N_{ev}$ $100pb^{-1}$ & $N_{ev}$ & $\varepsilon$ \\
%\hline
%\hline

%hlt (HE or VHE) &2466.4 $\pm$ 14.4& 30413.0 & 4.1e-01 $\pm$ 3.0e-03\\
%+ 2 ele (no Iso) pT $>30.0$ GeV &1357.5 $\pm$ 12.3& 13151.0 & 5.5e-01 $\pm$ 5.9e-03\\
%+ 2 ele (Iso) pT $>30.0$ GeV &1330.8 $\pm$ 12.2& 12851.0 & 9.8e-01 $\pm$ 1.3e-02\\
%+ 2e, 2j pT$_{ele}$(pT$_{jet}$) $>$30.0(50.0) GeV &124.9 $\pm$ 3.0& 1810.0 & 9.4e-02 $\pm$ 2.4e-03\\
%+ pT 1st ele $>$ 85.0 GeV &115.9 $\pm$ 2.9& 1697.0 & 9.3e-01 $\pm$ 3.2e-02\\
%+ Mee $<80.0$ GeV or Mee $>100.0$ GeV&16.2 $\pm$ 1.1& 235.0 & 1.4e-01 $\pm$ 1.0e-02\\
%+ St $>400.0$ GeV &8.2 $\pm$ 0.8& 120.0 & 5.0e-01 $\pm$ 5.8e-02\\
%\hline

%\end{tabular}
%\end{center}
%\caption{\small \sl $Z/\gamma$ sample: the first column lists the selection sequence, $N_{ev}$ $100pb^{-1}$ is the number of selected events in $100pb^{-1}$, $N_{ev}$ is the absolute number 
%of selected events (unweighted for the cross section), $\varepsilon$ is the relative selection efficiency with respect to the previous cut. 
%The overall selection efficiency is shown in the last line.}
%\label{tab:selection_effic_zjet}
%\end{table}

\begin{table}[htbp]
\begin{center}
\begin{tabular}{|cccc||ccc|}
\hline
                 &   Signal&         &          & & Background & \\
$M_{LQ}$=250 GeV & 400 GeV & 650 GeV & 1000 GeV & $t\bar{t}$  & $Z/\gamma$+jet & Others \\
\hline
425 $\pm$ 24 & 52.4 $\pm$ 2.5 & 3.4 $\pm$ 0.1 & 0.21 $\pm$ 0.01 & 21.3 $\pm$ 1.0 & 5.1 $\pm$ 0.6 & 1.1 $\pm$ 0.3 \\
\hline
\end{tabular}
\end{center}
\caption{\small \sl Number of selected events in the eejj sample expected in 100 pb$^{-1}$ of data for signal events at different LQ masses 
and for background events. 
The LQ cross section rapidly falls at high LQ mass, thus 
producing a relative decrease in the number of selected events. 
In the background part, ``Others'' includes $W$+jet, and 
di-boson ($WW$,$WZ$,$ZZ$) events. 
With the available MC statistics, no QCD multi-jet and $\gamma$+jet events pass the final selection.}
\label{tab:EventSelSummary}
\end{table}

%\begin{table}[htbp]
%\begin{center}
%\begin{tabular}{|c|c|c|c|}
%\hline
%\hline
% & $N_{ev}$ $100pb^{-1}$ & $N_{ev}$ & $\varepsilon$ \\
%\hline
%\hline

%+ LQ skim &1400.1 $\pm$ 14.0& 10000.0 & 1.0e+00 $\pm$ 1.4e-02\\
%+ 2 ele (no Iso) pT $>20.0$ GeV &793.3 $\pm$ 10.5& 5666.0 & 5.7e-01 $\pm$ 9.4e-03\\
%+ 2 ele (Iso) pT $>20.0$ GeV &689.4 $\pm$ 9.8& 4924.0 & 8.7e-01 $\pm$ 1.7e-02\\
%+ 2e, 2j pT$_{ele}$(pT$_{jet}$) $>$20.0(20.0) GeV &675.7 $\pm$ 9.7& 4826.0 & 9.8e-01 $\pm$ 2.0e-02\\
%+ ele, jet kinematics cuts &593.9 $\pm$ 9.1& 4242.0 & 8.8e-01 $\pm$ 1.8e-02\\
%+ Mee $<80.0$ GeV OR Mee $>100.0$ GeV&568.0 $\pm$ 8.9& 4057.0 & 9.6e-01 $\pm$ 2.1e-02\\
%+ St $>400.0$ GeV &562.4 $\pm$ 8.9& 4017.0 & 9.9e-01 $\pm$ 2.2e-02\\
%\hline

%full selection efficiency& &  & 4.0e-01 $\pm$ 7.5e-03\\
%\hline
%\end{tabular}
%\end{center}
% \caption{\small \sl $N_{ev}$ $100pb^{-1}$ is the number of selected events in $100pb^{-1}$, $N_{ev}$ is the absolute number of selected event (unweighted for the cross section), $\varepsilon$ is the relative selection efficiency with respect to the previous cut}
%\caption{\small \sl $N_{ev}$ $100pb^{-1}$ is the number of selected events in $100pb^{-1}$, $N_{ev}$ is the absolute number of selected event (unweighted for the cross section), $\varepsilon$ is the selection efficiency of each cut relatively to the sample selected by all the cuts above it.}
%\label{tab:selection_effic}
%\end{table}

% This is where I put figure~\ref{fig:Mej_allComb}.  I'm not sure if it's the correct place.
%INSERT SUMMARY TABLE WITH NUMBER OF BACKGROUND EVENTS. 




%Efficiencies (table of diff masses, bkgd, cut by cut)
%Number of selected events (for luminosity)

%\end{document}
