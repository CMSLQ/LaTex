%\documentclass{cmspaper}
%\begin{document}

%Fit method (histogram, unbinned, function, combination...)

\section{Signal Extraction} \label{sec:signalExtraction}

This section describes the procedure and verifies the ability to extract the 
number of signal events from a distribution of the invariant mass, $M_{ej}$, of the 
electron-jet pairs with the characteristics expected from the studies 
of the previous sections.

MC experiments are generated to produce $M_{ej}$ distributions that are the sum
of a leptoquark signal and the only relevant backgrounds: $t\bar{t}$ and $Z/\gamma$+jets.
The average leptoquark signal distribution, $h_s$, is taken from the fullSim MC 
sample at the leptoquark mass under consideration after applying the selection
described in section \ref{sec:eventSelection}.
The average distributions, $h_{t\bar{t}}$ and $h_{Z/\gamma\mathrm{+jets}}$ 
of the $t\bar{t}$ and $Z/\gamma$+jets backgrounds are determined from the control samples
and procedures described in section~\ref{sec:bkgStudy}.

A log-likelihood fit is performed on the histogram of each generated MC experiment
using a function
\begin{displaymath}
  h_{s+b} = N_s \cdot \tilde{h}_s + N_b \cdot \tilde{h}_b
\end{displaymath}
where $N_s$ and $N_b$ are the free parameters of the fit for the number of signal and background events,
\begin{displaymath}
  \tilde{h}_b = \frac{R}{1+R} \tilde{h}_{t\bar{t}} + \frac{1}{1+R} \cdot \tilde{h}_{Z/\gamma\mathrm{+jets}}~\mathrm{,}
\end{displaymath}
and $\tilde{h}_x$ represents a general distribution $h_x$ normalized to unity. 

The ratio
\begin{displaymath}
  R \equiv N_{t\bar{t}} / N_{Z/\gamma\mathrm{+jets}}~\mathrm{,}
\end{displaymath}
between the number of $t\bar{t}$ and $Z/\gamma$+jets events is currently determined from 
MC and has a value $R_{MC}=4.2$.
Figure~\ref{fig:Mej_fit} shows a MC experiment and the result of the fit.
A tendency of the fit to underestimate the number of signal events is observed. 
Such an effect amounts to about 10\% with a moderate dependency on the leptoquark mass and the 
integrated luminosity. This effect may bias the results towards an overestimation of the integrated
luminosity needed for discovery.

The fitting procedure described in this section is used in the following section
to determine the CMS potential for discovering a leptoquark signal. 

 \begin{figure}[htb]
   \begin{center}
     \resizebox{7cm}{!}{\includegraphics{plots/Mej_fit.eps}}
     \caption{\small \sl A MC experiment (dots with error bars) generated for a leptoquark mass of 
       400~GeV (blue-filled histogram), a sum of the $t\bar{t}$ and $Z/\gamma$+jets 
       backgrounds (gray-filled histogram) with a relative ratio R=4.2 determined from 
       full simulation MC, and a number of signal events $N_s=50$ corresponding
       to an integrated luminosity of 95~$pb^{-1}$.
       The blue-line open histogram is the result of the fit and the estimate of signal 
       events is 48.5$\pm$ 3.8.}
     \label{fig:Mej_fit}
   \end{center}
 \end{figure}


%Quality of fit?

%Pulls, etc...


%\end{document}
