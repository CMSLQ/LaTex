%\documentclass{cmspaper}
%\begin{document}
\section{Trigger Studies} \label{sec:trig}
% why I did it...
%All the trigger plots and explanations
%Quality of selection
%eff before and after analysis cuts, justification

The presence in the final state of two high energy electrons is exploited for the online selection 
of candidate leptoquark-antileptoquark events. 
Three electromagnetic High Level Trigger (HLT) paths have been investigated
 \footnote{In the most recent HLT trigger table these names have been changed to $HLT\_EM80$, $HLT\_EM200$, 
and $HLT\_IsoEle18\_L1R$.}
%In the most recent HLT trigger table these names have been chaged to HLT_EM80, HLT_EM200, and HLT_IsoEle18_L1R}
:
\begin{itemize}
\item the EMHighEt trigger 
\item the EMVeryHighEt  trigger
\item the SingleElectronRelaxed trigger.
\end{itemize}

The EMHighEt and EMVeryHighEt triggers have been specifically designed in order to have a high 
efficiency for electrons and photons with high transverse energy. 
Their input is the Level-1 relaxed single electromagnetic trigger.
EMHighEt triggers on a high transverse energy electromagnetic deposit (supercluster) with
$E_t>80~$GeV and loose isolation requirements (less than 5 GeV deposited in a 
cone of $\Delta R<0.3$ \footnote{$\Delta R=\sqrt{\Delta\eta^2 + \Delta\phi ^2}$} surrounding the 
primary supercluster in ECAL, less than 
8 GeV in cone of $0.15<\Delta R<0.3$ in HCAL and less than 4 tracks in a cone $\Delta R<0.3$ of the tracker).
EMVeryHighEt triggers on a very high transverse energy supercluster with $E_t>200~$GeV and no isolation 
requirements. SingleElectronRelaxed selects an electron with $E_t>18~$GeV, an HCAL over ECAL energy ratio less than 5\%, 
and $(1/E-1/p)<0.03$. More details on these triggers can be found in \cite{HeepHlt}.
% see https://twiki.cern.ch/twiki/bin/view/CMS/HEEPOnline
% https://twiki.cern.ch/twiki/bin/view/CMS/HighLevelTrigger06ExerciseDocumentation
% CMS AN 2007/009 and CMS Note 2006/078

The efficiencies for leptoquark events of the considered triggers have been studied with the FullSim 
samples of section~\ref{sec:MCSamples} at four leptoquark mass points and are reported in Table~\ref{tab:HLTeffic} and 
Fig.~\ref{fig:HLTeffic}.
% The efficiency of the OR of the 2 high energy triggers is high in the mass range ...
At a luminosity of $10^{32}~cm^{-1}s^{-1}$ the estimated rates of the EMHighEt, EMVeryHighEt and SingleElectronRelaxed triggers 
are 0.5, 0.1 and 10 Hz, respectively.
In order to guarantee a high efficiency and at the same time contain the data sample size, the current decision is to perform 
the analysis on the OR of the EMHighEt and EMVeryHighEt triggers.


  \begin{table}[htbp]
    \caption{\small \sl High Level Trigger efficiencies for LQ events at different mass. 
      Single triggers and their $OR$ are shown.}
    \label{tab:HLTeffic}
    \begin{center}
      \begin{tabular}{|l|cccc|} \hline
               & \multicolumn{4}{c|}{Leptoquark Mass (GeV)} \\
                                                     & 250 & 400 & 650 & 1000 \\ \hline
        EMHighEt                                     & 0.838$\pm$.012 & 0.933$\pm$.008 & 0.938$\pm$.005 & 0.944$\pm$.007   \\
        EMVeryHighEt                                 & 0.230$\pm$.013 & 0.648$\pm$.015 & 0.937$\pm$.005 & 0.990$\pm$.003   \\ 
        SingleElectronRelaxed                        & 0.805$\pm$.012 & 0.808$\pm$.013 & 0.765$\pm$.009 & 0.667$\pm$.015   \\ \hline
        EMHighEt OR EMVeryHighEt                     & 0.850$\pm$.011 & 0.963$\pm$.006 & 0.985$\pm$.002 & 0.994$\pm$.002   \\
        The above OR SingleElectronRelaxed           & 0.950$\pm$.007 & 0.979$\pm$.005 & 0.989$\pm$.002 & 0.995$\pm$.002   \\ \hline
      \end{tabular}
    \end{center}
  \end{table}

\begin{figure}[htbp]
  \begin{center}
  \begin{tabular}{cc}
  \resizebox{12cm}{!}{\includegraphics{plots/triggerEffic.eps}} 
  \end{tabular}
    \caption{\small \sl High Level Trigger efficiencies for LQ events at different mass. 
      Single triggers and their $OR$ are shown.}
    \label{fig:HLTeffic}
  \end{center}
\end{figure}




% add here the 1/30 of singleElecRelaxed?

%\end{document}

