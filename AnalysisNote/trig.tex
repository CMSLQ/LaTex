%\documentclass{cmspaper}
%\begin{document}
\section{Trigger Studies} \label{sec:trig}

The presence of two high energy electrons in the final state is exploited for the online selection 
of candidate leptoquark-antileptoquark events.
In order to be fully efficient to high energy electrons even with initial detector mis-alignment 
and calibration, triggers that require an high $E_T$ electromagnetic deposit without any isolation 
requirement and track-matching are considered.

Two trigger menus, named ``8E29'' and ``1E31'', have been studied and designed as candidates 
for the CMS start-up.
% More, if needed, at https://twiki.cern.ch/twiki/bin/view/CMS/TSG_18_II_09
In case that the ``8E29'' menu is used, the chosen trigger would be the ``HLT\_Photon15\_L1R''.
The corresponding Level-1 trigger seed is ``L1\_SingleEG8'', which requires an electromagnetic object with 
$E_T>8~$GeV. At the HLT level, the threshold applied is $E_T>15~$GeV and no isolation or track-matching
are required.
If the ``1E31'' menu is used, the above trigger will be prescaled, so the choice would shift to the
``HLT\_Photon25\_L1R'' trigger. The Level-1 trigger seed is still the ``L1\_SingleEG8''. At HLT level the 
threshold is $E_T>25~$GeV with no isolation or track-matching.

The efficiencies for leptoquarks for these triggers have been studied with the FullSim 
samples of Section~\ref{sec:MCSamples} at two leptoquark mass points, and have an inefficiency
of a fraction of a percent.
%% - trigger = HLT_Photon15_L1R
%% eff_LQ250 = 0.99571565802 +/- 0.00028628551
%% eff_LQ400 = 0.99715873016 +/- 0.00021206457
%% - trigger = HLT_Photon25_L1R
%% eff_LQ250 = 0.99342939481 +/- 0.00035412845
%% eff_LQ400 = 0.99630158730 +/- 0.00024184261







%\end{document}

