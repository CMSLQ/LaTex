%\documentclass{cmspaper}
%\begin{document}
\section{Trigger Studies} \label{sec:trig}

The presence of high energy electrons in the final state is exploited for the online selection 
of candidate leptoquark-antileptoquark events.
In order to be fully efficient to high energy electrons even with initial detector mis-alignment 
and calibration, triggers that require an high $E_T$ electromagnetic deposit without any isolation 
requirement and track-matching are considered.

Two trigger menus  (``low luminosity'' and ``high luminosity'' menus) 
have been studied and designed as candidates for 
two possible luminosity scenario of $8\times10^{29}$ cm$^{-2}$s$^{-1}$ and $10^{31}$ cm$^{-2}$s$^{-1}$,
respectively, expected for the LHC start-up.
% More, if needed, at https://twiki.cern.ch/twiki/bin/view/CMS/TSG_18_II_09
In case that the low luminosity menu is used, the chosen trigger would be the ``HLT\_Photon15\_L1R''.
The corresponding Level-1 trigger seed is ``L1\_SingleEG8'', which requires an electromagnetic object with 
$E_T>8~$GeV and relaxed ID requirements. 
At the HLT level, the threshold applied is $E_T>15~$GeV and no isolation or track-matching
are required. The expected HLT rate of this trigger is around 10 Hz for the low luminosity scenario.
If the high luminosity menu is used, the above trigger will be prescaled, so the choice would shift to the
``HLT\_Photon25\_L1R'' trigger. The Level-1 trigger seed is still the ``L1\_SingleEG8''. At HLT level the 
threshold is $E_T>25~$GeV with no isolation or track-matching. 
The expected HLT rate of this trigger is around 20 Hz for the high luminosity scenario.

The efficiencies for leptoquarks for these triggers have been studied with the FullSim 
samples of Section~\ref{sec:MCSamples} at two leptoquark mass points, and have efficiencies approaching
100\% as shown in Table~\ref{tab:HLTEffic}.

\begin{table}[htbp]
\begin{center}
\begin{tabular}{|c|c|c|c|}
\hline\hline
$M_{LQ}$ (GeV)     & ``HLT\_Photon15\_L1R''  &  ``HLT\_Photon25\_L1R'' \\
\hline\hline
250                & 99.6\%   & 99.3\%  \\
400                & 99.7\%   & 99.6\%  \\
%250               & 0.99571  $\pm$ 0.00029  & 0.99343  $\pm$ 0.00035   \\
%400               & 0.99716  $\pm$ 0.00021  & 0.99630  $\pm$ 0.00024   \\
\hline\hline
%% - trigger = HLT_Photon15_L1R
%% eff_LQ250 = 0.99571565802 +/- 0.00028628551
%% eff_LQ400 = 0.99715873016 +/- 0.00021206457
%% - trigger = HLT_Photon25_L1R
%% eff_LQ250 = 0.99342939481 +/- 0.00035412845
%% eff_LQ400 = 0.99630158730 +/- 0.00024184261
\end{tabular}
\end{center}
\caption{Efficiencies of the chosen HLT triggers for LQ masses of 250 and 400~GeV (FullSim samples are used). L1 efficiencies are included.
Relative statistical uncertainties on the trigger efficiencies are less than 0.1\%.}
\label{tab:HLTEffic}
\end{table}

% FIXME add trigger rates - but could not find them in the twiki https://twiki.cern.ch/twiki/bin/view/CMS/TSG_18_II_09





%\end{document}

