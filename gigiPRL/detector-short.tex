\documentclass[12pt]{article}
\begin{document}

\begin{center}
{\large
\textbf{Inputs for the ``detector description paragraph''\\ of the CMS physics papers}}

\medskip
Please send any comments you may have to Carlos Louren\c{c}o
\end{center}

\medskip

\emph{This file contains a few paragraphs and some isolated sentences
  that should be taken as starting points to prepare the ``detector
  description'' paragraph(s) to be included in the CMS physics
  publications (letters or papers).}

\emph{It is assumed that the papers start with an introductory
  paragraph describing the physics motivation of the paper, followed
  by the short detector description we deal with in here.  In general,
  we should always start with the sentence:}

A detailed description of the Compact Muon Solenoid (CMS) experiment
can be found elsewhere~\cite{JINST}.

\emph{Where ``\cite{JINST}'' stands for the CMS detector paper,
  published in JINST.  This is to be followed by a concise overview of
  the most important elements of the CMS experiment, such as:}

The central feature of the CMS apparatus is a superconducting
solenoid, of 6~m internal diameter.  Within the field volume are the
silicon pixel and strip tracker, the crystal electromagnetic
calorimeter (ECAL) and the brass-scintillator hadronic calorimeter
(HCAL).  Muons are measured in gas chambers embedded in the iron
return yoke.  Besides the barrel and endcap detectors, CMS has
extensive forward calorimetry.

\emph{Finally, the paragraph should end with a few lines oriented
  towards the detectors essential to the analysis presented in each
  specific paper.  For instance, a paper on jets should mention the
  ECAL and HCAL energy resolutions and give a few geometrical
  details:}

The ECAL has an energy resolution of better than 0.5\,\% above
100~GeV.  The HCAL, when combined with the ECAL, measures jets with a
resolution $\Delta E/E \approx 100\,\% / \sqrt{E} \oplus 5\,\%$.  The
calorimeter cells are grouped in projective towers, of granularity
$\Delta \eta \times \Delta \phi = 0.087\times0.087$ at central
rapidities and $0.175\times0.175$ at forward rapidities.

\emph{A paper where the muons play a central role would, instead, give
  emphasis to the muon measurement:}

The muons are measured in the pseudorapidity window $|\eta|< 2.4$,
with detection planes made of three technologies: Drift Tubes, Cathode
Strip Chambers, and Resistive Plate Chambers.  Matching the muons to
the tracks measured in the silicon tracker results in a transverse
momentum resolution between 1 and 5\,\%, for $p_{\rm T}$ values up to
1~TeV/$c$.

\emph{Papers on topics where charged particle tracking is crucial,
  such as a paper on charged particle multiplicities and spectra,
  should include more details on the silicon tracker, such as:}

Mid-rapidity charged particles are tracked by three layers of silicon
pixel detectors, made of 66~million $100\times150$~$\mu$m$^2$ pixels,
followed by ten microstrip layers, with strips of pitch between 80 and
180~$\mu$m.

\emph{Papers where vertexing is crucial should also mention the
  vertexing resolution, as appropriate.  For instance:}

The silicon tracker provides the vertex position with
$\sim$\,15~$\mu$m accuracy.

\emph{Papers where the trigger system needs to be mentioned, could
  include something along the following lines:}

The first level (L1) of the CMS trigger system, composed of custom
hardware processors, uses information from the calorimeters and muon
detectors to select (in less than 1~$\mu$s) the most interesting
events (only one bunch crossing in 1000).  The High Level Trigger
(HLT) processor farm further decreases the event rate from 100~kHz to
100~Hz, before data storage.

\emph{Other papers, devoted to forward physics, would need to detail
  the existence of forward detectors, with sentences of the kind:}

The very forward angles are covered by the CASTOR ($5.3<|\eta|<6.6$)
and Zero Degree ($|\eta|>8.3$) calorimeters.

Two extra tracking stations, built by the TOTEM experiment, are placed
at forward rapidities ($3.1<|\eta|<4.7$ and $5.5<|\eta|<6.6$).

\emph{Depending on the paper, other information could be provided, if
  deemed useful, such as:}

The CMS apparatus has an overall length of 22~m, a diameter of 15~m,
and weighs 12\,500 tonnes.

The lead-tungstate crystals are 25.8\,$X_0$ thick in the barrel
and 24.7\,$X_0$ thick in the end-caps.

The muon detection system has nearly 1 million electronic channels.

The ZDC and CASTOR calorimeters are made of quartz fibers/plates
embeded in tungsten absorbers.

In total, there are $\sim$\,$10^8$ data channels checked in each bunch
crossing.

\begin{thebibliography}{9}
\bibitem{JINST} CMS Collaboration, CMS Detector Paper, JINST (in print).
\end{thebibliography}

\end{document}
